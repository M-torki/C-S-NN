\begin{center}
	\begin{minipage}{\textwidth}
		\begin{traditionalpoem}
			این چرخِ فلک که ما در او حیرانیم، & 
			
			فانوس خیال
			\footnotemark
			از او مثالی دانیم: \\
			
			خورشیدْ چراغ‌دان و عالَم فانوس،&
			
			ما چون صُوَریم کاندر او گَردانیم \\
		\\
       & 
       			\hspace{45pt}
       			خیام
		\end{traditionalpoem}
	\end{minipage}
\end{center}
\footnotetext{
	فانوس خيال، وسیله‌ای بسیار قدیمی است كه هنوز هم در هندوستان به كار مي رود.  فانوس یا شمع درون استوانه‌ای  که نقوشی  روی آن حک شده بودند قرار می‌گرفت. استوانه به قدري سبك بود که هوا در آن جريان داشت و دور شمع می‌گشت. در نتیجه سایه تصاویر روی دیوار حرکت می‌کرد.}

\section{مدل استاندارد کیهان‌شناسی}
\label{sec:lcdm}


چیستی و چرایی و چگونگی کیهان همواره در طول تاریخ از سوالات بنیادین بشریت بوده است و لذا کیهان‌شناسی از علوم کهنی است که قدمتی به اندازه عمر بشریت دارد. اما تعریف کیهان و به تبع آن کیهان‌شناسی همواره یکسان نبوده است. واژه یونانی 
\lr{$\kappa$o$\sigma\mu$os}
 در برابر $chaos$، به معنی نظم و منظم‌شده است. 
 \gu{این نام‌گذاری نزد یونانیان این مفهوم را به همراه داشت که عالم کائنات الگوی نظمی است که هم زیباست و هم موجودی زنده و ذی‌شعور و به بیان افلاطونی، متعالی‌ترین موجود محسوس است.}(سید حسین نصر، دین و نظم طبیعت، ترجمه محمد فغفوری، انتشارات حکمت، تهران، چ۱، ۱۳۸۴، ص۲) کیهان‌شناسی به معنای سنتی کلمه، دانش مابَعدالطبیعه را نیز در دل خود داراست و از فلسفه غیرقابل تمییز است. اما آن‌چه امروز کیهان‌شناسی نام گرفته است، دانشی است تجربی که پس از کشف نظریه نسبیت عام اینشتین در اوایل قرن بیستم میلادی، برای اولین بار در تاریخ به عنوان یک علم کمّی و آزمایش‌پذیر شناخته می‌شود. \\ 
 کیهان‌شناسی مدرن علم بررسی کیهان به عنوان یک کل است، به عبارتی در مقیاس‌های کیهانی تنها خوشه‌های کهکشانی را می‌بینیم البته فقط به عنوان یک سری نقطه و خبری از ستاره‌ها و منظومه‌ها و... نیست. در این بخش مشاهدات بنیادی اخیر و پس از آن مدل استاندارد کیهان‌شناسی را با تکیه بر مرجع 
\cite{piattella2018lecture}  
 معرفی می‌کنیم.     
نقطه شروع ما در مطالعه کیهان‌شناسی یک مشاهده شگفت‌انگیز است: کیهان در حال انبساط است. ادوین هابل
\LTRfootnote{Edwin Hubble}
کشف کرد که کهکشان‌های دورتر با سرعت بیشتری از ما دور می‌شوند. 
\cite{hubble1929relation}
این بیان قانون مشهور هابل است که رابطه بین سرعت و فاصله را نشان می‌دهد:
\begin{equation}
v = H_0r
\end{equation}
$H_0$
ثابت هابل است که نرخ انبساط با آن نشان داده می‌شود. مقداری که هابل برای نرخ انبساط از روی نمودار شکل
\ref{fig:v-r rel}
به دست آورده بود با توجه به محدودیت در اندازه‌گیری‌ها خطای زیادی داشت. اخیراً در مقاله مرجع  
\cite{gil2016clustering} 
تحت پروژه eBOSS 
\LTRfootnote{The Extended Baryon Oscillation Spectroscopic Survey}
برای ثابت هابل مقدار  	
\begin{center}
$H_0 = 67.6 \pm 0.7 \;  km\; s^{-1} \; Mpc^{-1}$
\end{center}
گزارش شده است.


\begin{figure}
	\begin{center}
		\includegraphics[scale=0.3]{figs/r-v_rel.png}
	\end{center}
	\caption[
	نمودار سرعت-فاصله برای سحابی‌های فراکهکشانی، مربوط به مقاله اصلی هابل سال ۱۹۲۹ مرجع 
	\cite{hubble1929relation} 
	]{نمودار سرعت-فاصله برای سحابی‌های فراکهکشانی، مربوط به مقاله اصلی هابل سال ۱۹۲۹ مرجع 
		\cite{hubble1929relation}
	
    }
	\label{fig:v-r rel}
\end{figure}
در اثر انبساط کیهان پرتوهای دریافتی‌ از اجرام، قرمزتر از پرتوهای گسیل‌شده هستند. با توجه به این مسئله، قرمزگرایی
\LTRfootnote{Redshift}
را به عنوان یکی از مشاهده‌پذیر‌های بنیادی در کیهان‌شناسی به این صورت تعریف می‌کنیم:
\begin{equation}
z = \frac{\lambda_{obs}}{\lambda_{em}} -1
\end{equation}
\label{eq:z}
که در این رابطه  $\lambda_{obs}$ طول موج پرتو دریافتی و $\lambda_{em}$ طول موج گسیل‌شده از چشمه است.
\par
یکی دیگر از کشفیات تاثیرگذار در قرن اخیر که کیهان‌شناسی را وارد دوره جدیدی کرد به وسیله کشف ابرنواخترهای نوع Ia  
\cite{williams1996hubble} 
در سال ۱۹۹۷ اتفاق افتاد. تحلیل رفتار این نوع ابرنواخترها منجر به این نتیجه شد که کیهان در فاز انبساط تندشونده است. 
\cite{perlmutter1999measurements , riess1998observational}  
این کشف مهم جایزه نوبل فیزیک ۲۰۱۱ را به خود اختصاص داد. مسئله انبساط شتاب‌دار این سوال را به وجود می‌‌آورد که چه چیز منجر به تند شدن انبساط می‌شود، مگر نه اینکه خاصیت گرانش این است که ماده را جذب و در نتیجه انبساط را کُند کند؟  یکی از راه‌حل‌های پیشنهادی برای توضیح این مشاهده، وجود نوع جدیدی از ماده یا انرژی است که خاصیت پاد-گرانشی داشته باشد که به نام انرژی تاریک
\LTRfootnote{Dark Energy} 
شناخته می‌شود. یکی از کاندیدهای انرژی تاریک وجود یک ثابت در معادلات توصیف‌کننده کیهان است که به ثابت کیهان‌شناسی $\Lambda$ معروف است. پیش از کشف انبساط کیهان، اینشتین برای آن‌که معادلاتش به کیهانی ایستا( مطابق شهودش در آن زمان) بیانجامد یک ثابت به معادلاتش افزود اما پس از کشف هابل این ثابت را از معادلات خود حذف کرد و آن را بزرگترین اشتباه خود
\LTRfootnote{Einstein's biggest blunder}
خواند. اما دیری نپایید که این ثابت به عنوان توجیه انبساط شتاب‌دار مجددا به معادلات بازگشت.
\par
شواهد فراوانی در کیهان نشان می‌دهند که بخش تاریک کیهان به انرژی تاریک خلاصه نمی‌شود. مولفه تاریک دیگر، ماده تاریک
\LTRfootnote{Dark Matter}
است که به گونه‌ای از ماده گفته می‌شود که برهم‌کنش الکترومغناطیس ندارد و مانند ماده معمولی دیده نمی‌شود و تنها از روی اثرات گرانشی می‌توان آن را مشاهده کرد. پیشنهاد وجود ماده تاریک در ابتدا توسط زوییکی
\LTRfootnote{Fritz Zwicky}
در سال ۱۹۳۳ مطرح شد.
شواهدی که بر وجود ماده تاریک صحّه ‌گذاشتند عبارتند از:
\begin{itemize}
	\item 
	دینامیک کهکشان‌ها در خوشه‌ها: به طور مثال جرم ویریال برای خوشه کهکشانی کُما
	\LTRfootnote{Coma cluster}
	، ۵۰۰ برابر جرمی است که از روی مشاهدات نوری تخمین زده می‌شود.
	\cite{zwicky1933rotverschiebung}
	
	\item
	منحنی‌های چرخش کهکشان‌های مارپیچی: منحنی سرعت چرخش ستاره‌هایی که در قسمت‌های خارجی کهکشان‌های مارپیچی قرار دارند بر خلاف تصور کپلری، صاف می‌شود در حالی که انتظار داریم سرعت در خارج از برآمدگی کهکشان متناسب با معکوس جذر فاصله از مرکز کهکشان کاهش پیدا کند.    
	\cite{sofue1999central , sofue2001rotation}
 	\item
 	تشکیل ساختارها در عالم: اگر ماده تاریک وجود نداشت چگالی ماده باریونی کنونی موجود در کیهان بسیار کمتر از مقداری می‌شد که امروزه مشاهده می‌کنیم و غیرخطی بودن این چگالی توجیه نمی‌شد.
 	
\end{itemize}
و دلایل بسیار دیگری که در اینجا به آن‌ها اشاره نمی‌کنیم. کاندیدهای مشهوری که برای ماده تاریک وجود دارد از ورای مدل استاندارد ذرات بنیادی می‌آیند. مشهورترین آن‌ها ذرات جرم‌دار با برهم‌کنش ضعیف
\LTRfootnote{Weakly Interacting Massive Particles}
یا WIMP است.
\cite{pospelov2008secluded}
با توجه به شواهد موجود، علاوه بر اینکه نتیجه می‌گیریم که ماده تاریک باید وجود داشته باشد، انتظار داریم که سرد نیز باشد. سرد بودن به معنی سرعت پایین ذرات و فشارِ قابل صرف‌نظر کردن است. خواننده علاقه‌مند در مرجع 
\cite{bertone2018history}
می‌تواند درباره ماده تاریک و کاندیدهای آن بیشتر بخواند.
\par
نظریه ثابت کیهان‌شناسی $\Lambda$ در کنار ماده تاریک سرد یا CDM در کنار هم مدل استاندارد کیهان‌شناسی 
$\Lambda CDM$
را می‌سازند که با انضمام نظریه تورم
\LTRfootnote{Inflation}
تا کنون موفق‌ترین و متداول‌ترین نظریه در جامعه فیزیک برای توصیف رفتار بزرگ‌مقیاس کیهان بوده است.
 
\subsection{عامل مقیاس و معادلات فریدمن}
\label{subsec:scale_factor}

\begin{figure}
	\begin{center}
		\includegraphics[scale=0.3]{figs/scalef.png}
	\end{center}
	\caption[
	انبساط کیهان و عامل مقیاس. مختصه همراه $x_1$ و $x_2$  دارای فاصله همراه ثابت در طول زمان هستند، این فاصله ۱ در نظر گرفته شده است. ولی فاصله فیزیکی آن‌ها در طول زمان افزایش می‌یابد که مقدار این افزایش متناسب با عامل مقیاس 
	$a(t)$ 
	در هر زمان است. شکل از مرجع 
	\cite{dodelson2020modern}]
	{ 
		انبساط کیهان و عامل مقیاس. مختصه همراه $x_1$ و $x_2$ دارای فاصله همراه ثابت در طول زمان هستند، این فاصله ۱ در نظر گرفته شده است. ولی فاصله فیزیکی آن‌ها در طول زمان افزایش می‌یابد که مقدار این افزایش متناسب با عامل مقیاس
		$a(t)$ 
		در هر زمان است. شکل از مرجع  
		\cite{dodelson2020modern}
	}
	\label{fig:scalef}
\end{figure}

گفتیم که کیهان در حال انساط است، یعنی فاصله بین ما و کهکشان‌های بسیار دور بیشتر از قبل شده است. می‌توانیم این افزایش طول را با یک عامل مقیاس
\LTRfootnote{scale factor}
توصیف کنیم. فرض کنید نقاط در کیهان روی یک شبکه قرار بگیرند و مختصات هر نقطه ثابت باشد. مطابق شکل 
\ref{fig:scalef}
اگر این شبکه منبسط شود مختصات نقاط که به آن مختصه همراه می‌گوییم، دست نمی‌خورد و در عین حال فاصله نقاط از هم بیشتر می‌شود. عامل مقیاس را در زمان حال ۱ در نظر می‌گیریم و هر چه در زمان به عقب می‌رویم این مقیاس کوچک‌تر می‌شود.

با این تعریف از عامل مقیاس می‌توانیم قرمزگرایی رابطه 
\ref{eq:z}
را بازنویسی کنیم. از آن‌جایی که طول موج متناسب با عامل مقیاس کش می‌آید:
\begin{equation}
	z = \frac{\lambda_{obs}}{\lambda_{em}} -1 = \frac{a_{obs}}{a_{em}} -1 = \frac{1}{a_{em}} -1
\end{equation}
\label{eq:z_a}

بررسی تحولات عامل مقیاس در طول زمان می‌تواند در مورد تاریخچه کیهان به ما اطلاعاتی دهد. علاوه بر عامل مقیاس، هندسه کیهان نیز پارامتر دیگری است که لازم است بدانیم. برای هندسه کیهان سه حالت وجود دارد: باز، تخت و بسته. نسبیت عام هندسه را به انرژی مرتبط می‌کند: اگر چگالی بیش از یک مقدار بحرانی باشد کیهان بسته و اگر کم‌تر از چگالی بحرانی باشد باز خواهد بود. هندسه تخت یا اقلیدسی نیز در صورتی اتفاق می‌افتد که چگالی انرژی برابر مقدار بحرانی باشد. 
برای بررسی تحولات کیهان باید متریک
\LTRfootnote{metric}
توصیف کننده کیهان را داشته باشیم. برای به دست آوردن متریک به مشاهدات کیهانشناختی رجوع می‌کنیم: کیهان در مقیاس‌های بزرگ، توزیع خوشه‌های کهکشانی همسانگرد 
\LTRfootnote{isotropic}
است. به هر جهتی از کیهان که بنگریم یک شکل است. مشاهده دیگر در کیهان بزرگ مقیاس، همگن 
\LTRfootnote{homogeneous}
بودن کیهان است. اگر کیهان حول تمام نقاط همسانگرد باشد این به معنی همگنی است. فرض همگنی و همسانگردی کیهان به عنوان اصل کیهان‌شناختی
\LTRfootnote{cosmological principle}
 شناخته می‌شود.
\cite{baumann2014cosmology}  
\begin{figure}[h!]
	\begin{center}
		\includegraphics[scale=0.3]{figs/gal_dist.png}
	\end{center}
	\caption{ 
  توزیع کهکشان‌‌ها در مقیاس‌های کوچک به شکل توده‌ شده است در حالی که در مقیاس‌های بزرگتر و در زمان‌های اولیه کیهان همگنی و همسانگردی به وضوح دیده می‌شود. شکل از 
		\cite{baumann2014cosmology}
	.}
	\label{fig:gal_dist}
\end{figure}
مطابق شکل
\ref{fig:gal_dist}
هر چه در عالم به نقاط دوردست و مقیاس‌های بزرگتر با z بیشتر نگاه می‌کنیم کیهان یک‌دست‌تر می‌شود. اصل کیهانشناختی، یا همان فرض همگنی و همسانگردی یک فرم یکتا از هندسه فضا-زمان را نتیجه می‌دهد. توزیع ماده و انرژی در عالم وابستگی متریک به مکان در فضا-زمان را مشخص می‌کند. بدون شناختن ماده نمی‌توانیم متریک را مشخص کنیم ولی اصل همگنی و همسانگردی تقارن‌هایی را وارد می‌کند که مسئله را ساده‌تر می‌کنند. می‌توانیم بخش فضایی متریک را به عنوان ۳-فضاهایی با تقارن حداکثری در نظر بگیریم. متریک توصیف کننده این فضا برای حالتی که عالم در حال انبساط است متریک فریدمن-رابرتسون-واکر
\LTRfootnote{Friedmann-Robertson-Walker}
 یا FRW است که به شکل 
 \begin{equation}
 {ds}^2 = {dt}^2 - a^2(t) [ \frac{{dr}^2}{1-kr^2}	+ r^2 {d\Omega}^2]
 \end{equation} 
    است. در این متریک $k$ عددی ثابت و نشان‌دهنده هندسه فضایی است. $k$  سه مقدار $ ۰،۱ و ۱- $رابه خود می‌گیرد که به ترتیب مربوط به هندسه کروی، اقلیدسی یا تخت و هذلولوی است. حال که با متریک یک کیهان در حال انبساط همگن و همسانگرد آشنا شدیم به سراغ دینامیک کیهان می‌رویم و برای این مبحث از منبع 
\cite{dodelson2003modern}
استفاده می‌کنیم. مطابق چیزی که از نسبیت عام می‌دانیم، معادله اینشتین رابطه بین هندسه و ماده را به ما می‌دهد. معادله اینشتین به شکل
 \begin{equation}
G_{\mu\nu} \equiv R_{\mu\nu} - \frac{1}{2} g_{\mu\nu} \mathcal{R} = 8\pi G T_{\mu\nu}
\end{equation}     
است که در اینجا 
$G_{\mu\nu}$
تانسور اینشتین، 
$R_{\mu\nu}$
تانسور ریچی،
$\mathcal{R}$
اسکالر ریچی و 
$T_{\mu\nu}$
تانسور انرژی-تکانه است. 
تانسور ریچی را با استفاده از نماد کریستوفل تعریف می‌کنیم:
\begin{equation}
R_{\mu\nu} = \Gamma^{\alpha}_{\mu \nu , \alpha} - \Gamma^{\alpha}_{\mu \alpha , \nu}
+ \Gamma^{\alpha}_{\beta , \alpha} \Gamma^{\beta}_{\mu \nu}
- \Gamma^{\alpha}_{\beta , \nu} \Gamma^{\beta}_{\mu \alpha}
\end{equation}  
که ویرگول در رابطه به معنای مشتق جزئی نسبت به $x$ است. یعنی
\begin{equation}
\Gamma^{\alpha}_{\mu \nu , \alpha} = \frac{\partial \Gamma^{\alpha}_{\mu \nu } }{\partial x^{\alpha}}
\end{equation} 
و نماد کریستوفل نیز اینگونه تعریف می‌شود: 
\begin{equation}
\Gamma^{\mu}_{\alpha \beta } = \frac{g^{\mu \nu}}{2} [\frac{\partial g_{\alpha \nu}}{\partial x^{\beta}}
+ \frac{\partial g_{\beta \nu}}{\partial x^{\alpha}}
- \frac{\partial g_{\alpha \beta}}{\partial x^{\nu}}]
\end{equation} 

با توجه به این تعاریف مقدار 
$R^{00}$
 را می‌توانیم به دست آوریم. با توجه به اینکه تنها مولفه‌هایی از کریستوفل صفر نیستند که  
$ \mu = \nu = 0$
یا
$ \mu = \nu = i$
پس $R^{00}$ به دست می‌آید:
\begin{flalign}
\begin{aligned}
&R^{00} = \Gamma^{\alpha}_{00 , \alpha} - \Gamma^{\alpha}_{0 \alpha , 0} + \Gamma^{\alpha}_{\beta \alpha} \Gamma^{\beta}_{00} - \Gamma^{\alpha}_{\beta 0} \Gamma^{\beta}_{0\alpha} &\\
& \; \; \; \; \;= - \Gamma^{i}_{0i , 0}  - \Gamma^{i}_{j0} \Gamma^{j}_{0i} &\\
& \; \; \; \;\;= -3[\frac{\ddot{a}}{a} - \frac{\dot{a}^2}{a^2}] - 3(\frac{\dot{a}}{a})^2  & \\
& \; \; \;  \;\;= -3\frac{\ddot{a}}{a} &
\end{aligned}&&& 
\end{flalign} 
به روش مشابه میتونیم مولفه‌های $R_{ij}$ را نیز به دست آوریم:
\begin{equation}
R_{ij} = \delta_{ij} [2\dot{a}^2 + a\ddot{a}]
\end{equation}  
اسکالر ریچی نیز با ادغام تانسور ریچی اینگونه حاصل می‌شود:
\begin{flalign}
\begin{aligned}
&\mathcal{R} = g^{\mu \nu} R_{\mu \nu} = -R_{00} + \frac{1}{a^2} R_{ij} & \\
& \; \; \; = 6 [\frac{\ddot{a}}{a} + (\frac{\dot{a}}{a})^2] &
\end{aligned}&&& 
\end{flalign} 
حال که اسکالر و تانسور ریچی را به دست آوردیم، به معادله اینشتین برمیگردیم و مولفه $00$ آن را می‌نویسیم.
\begin{equation}
R_{00} - \frac{1}{2} g_{00} \mathcal{R} = 8\pi G T_{00}
\end{equation} 
با توجه به اینکه مولفه زمان-زمان تانسور انرزی-تکانه همان چگالی انرژی یا $\rho$ است با جایگذاری مقادیر ریچی به معادله اول فریدمن می‌رسیم:
\begin{equation}
(\frac{\dot{a}}{a})^2  = \frac{8 \pi G}{3} \rho
\label{eq:fried}
\end{equation} 

که معادله فریدمن تحول عامل مقیاس را با توجه به چگالی مشخص می‌کند. سمت چپ معادله همان مجذور نرخ هابل است و با توجه به تعریف چگالی بحرانی 
\begin{equation}
\rho_{cr} \equiv \frac{3 H_0^2 }{8 \pi G}
\end{equation} 
مقداری را برای چگالی مشخص می‌کند که اگر چگالی آن مقدار باشد کیهان تخت می‌شود، معادله فریدمن را می‌توانیم به صورت معادل زیر بنویسیم.
\begin{equation}
\frac{H^2(t)}{H_0^2} =  \frac{\rho}{\rho_{cr}}
\end{equation}
در اینجا چگالی انرژی $\rho$ سهم مربوط به همه اقسام ماده، تابش و انرژی تاریک را شامل می‌شود. و البته ما در اینجا کیهان را تخت فرض کرده و از سهم انحنای عالم صرف‌نظر کرده‌ایم. 
بقای انرژی که به معنی  
$\nabla T×{\mu}ـ{\nu} = 0 $
است تضمین می‌کند که
\begin{equation}
\dot{\rho} = -3(\rho + P) (\frac{\dot{a}}{a})
\label{eq:cont}
\end{equation} 

با استفاده از معادله 
\ref{eq:fried}
و هم‌چنین
\ref{eq:cont}
و با تکیه بر معادله حالت 
 \begin{equation}
w = \frac{P}{\rho} = constant.
 \end{equation} 
 می‌توانیم به معادله تحول چگالی انرژی برای اقسام مختلف ماده و انرژی برسیم:
 \begin{equation}
\rho = \rho_0 (\frac{a_0}{a})^{3(1+w)}
\end{equation}  
که در اینجا 
$\rho_0$
و
$a_0$
به مقدار چگالی انرژی و عامل مقیاس در زمان کنونی اشاره دارد. $w$ برای مولفه‌های مختلف کیهان مقادیر متفاوتی دارد. برای ماده غیرنسبیتی ۰، برای تابش $\frac{1}{3}$ و برای انرژی تاریک مقدار ۱- را داراست. پس چگالی انرژی در طول انبساط کیهان برای مولفه‌های مختلف با سرعت‌های متفاوتی افت می‌کند. چگالی انرژی برای ماده غیرنسبیتی با
 $a^{-3}$
 و مولفه تابش با 
  $a^{-3}$
  تغییر می‌کند در حالی که برای مولفه انرژی تاریک، چگالی انرژی در طول انبساط کیهان ثابت است. در تاریخچه تحول کیهان در ابتدا یک فاز تابش غالب وجود داشته است ولی از آنجایی که چگالی تابش با سرعت بیشتری نسبت به سایر مولفه‌ها افت داشته است، لذا با کم شدن چگالی انرژی تابش، دوران تابش غالب پایان یافته و کیهان وارد فاز ماده غالب شده است. انرژی تاریک اما رفتار غیرمتعارفی دارد. انتظار ما از ماده معمولی این است که چگالی انرژی آن با انبساط کیهان کاهش پیدا کند و رقیق شود. انرژی تاریک اما چگالی ثابتی در طول زمان و انبساط کیهان دارد. مولفه‌های دیگر یعنی ماده و تابش در اثر انبساط رقیق می‌شوند در حالی که چگالی انرژی تاریک دست نخورده باقی مانده است و لذا زمانی فرا می‌رسد که چگالی سایر مولفه‌ها به قدری کم شود که کیهان را وارد فاز انرژی غالب بکند. معادله فریدمن را می‌توان با تقسیم کردن بر مقدار چگالی بحرانی به شکل دیگری نوشت:
\begin{flushleft}
	\begin{equation}
H^2(t) = H_0^2 ( \Omega_r a^{-4} + \Omega_m a^{-3}+ \Omega_k a^{-2}+ \Omega_{\Lambda} )
\end{equation}
\end{flushleft}

\begin{figure}[h!]
	\begin{center}
		\includegraphics[scale=0.35]{figs/6param.png}
	\end{center}
	\caption{ مقادیر بهترین برازش برای ۶ پارامتر آزاد مدل استاندارد کیهان‌شناسی، به دست آمده از داد‌ه‌های پلانک ۲۰۱۸.
	\cite{akrami2018legacy}}
	\label{fig:6param}
\end{figure}
  

\section{تاریخچه گرمایی کیهان و سرگذشت فوتون‌ها }
\label{subsec:thermal}
مدل آشنای مهبانگ
\LTRfootnote{Big Bang}
پیشنهاد می‌کند که کیهان از یک حالت چگال و بسیار داغ آغاز می‌شود. در ابتدا که دما بسیار بالاست یک سوپ کیهانی داریم که  تمام اجزای آن در تعادل اند و نرخ اندرکنش‌ها بسیار بیشتر از نرخی است که در زمان حال شاهد آن هستیم. به طور مثال طول پویش آزاد یک فوتون در زمان کنونی 
$10^{26} m$
است در حالی که طول پویش آزاد برای فوتون در ۱ ثانیه ابتدای کیهان تقریبا به اندازه طول یک اتم بوده است. 
\cite{dodelson2003modern}
یک فوتون در زمان حال می‌تواند در کیهان پیمایش کند بدون اینکه دچار برهم‌کنش زیادی شود. اگر تمام اجزا در تعادل باقی می‌ماندند و هیچ کدام از ساختارهایی که امروزه می‌بینیم شکل نمی‌گرفتند، پس فرآیندهای خارج از تعادل نقش مهمی در کیهان‌شناسی دارند. علت خارج شدن ذرات از تعادل، پایین آمدن دما در اثر انبساط کیهان است. یکی از اجزایی که از تعادل خارج شده است، فوتون‌هایی هستند که در حدود ۳۸۰۰۰۰ سالگی کیهان از سوپ کیهانی جدا شده اند و پس از آن دیگر با ماده برهم‌کنش نداشته اند. آسمان گرچه تاریک به نظر می‌رسد اما اگر با ابزار مناسبی به آن بنگریم از همه جهت‌ها تابشی دریافت خواهیم کرد که دمای ۲/۷ درجه کلوین دارد. این فوتون‌ها همان فوتون‌های بازمانده از مهبانگ هستند به ما می‌رسند. . تابش زمینه کیهان
\LTRfootnote{Cosmic Microwave Background}
، که فوتون‌های بازمانده از مهبانگ هستند اولین بار در سال ۱۹۶۵ توسط آرنو پنزیاس 
\LTRfootnote{Arno Penzias}
و رابرت ویلسون
\LTRfootnote{Robert Woodrow Wilson}
مشاهده شدند. 
\cite{penzias1965measurement} 
آشکارسازی تابش زمینه کیهانی برای توسط این دو فرد به صورت تصادفی و در حالی که به دنبال مطالعه اثرات رادیویی ماهواره‌های مخابراتی بودند برای اولین بار انجام گرفت و جایزه نوبل ۱۹۷۸ را نیز از آن خود کردند. گروه پیبلز 
\LTRfootnote{James Peebles}
و همکاران نیز به صورت هدفمند به دنبال این آشکارسازی این تابش بودند که آن‌ها نیز رصد این تابش را توسط پنزیاس و ویلسون تایید کردند. 
 \cite{dicke1965cosmic}
 گفتنی است که پیبلز نیز در نهایت با چهل و اندی سال تاخیر به خاطر کشفیات نظری در کیهان‌شناسی برنده جایزه نوبل سال ۲۰۱۹ شد!
تابش زمینه کیهانی یکی از مهم‌ترین شواهد برای مدل استاندارد کیهان‌شناسی است که پیش‌تر توسط گاموف 
\LTRfootnote{George Gamow}
و همکاران پیش‌بینی و دمای آن حدودا ۵ درجه کلوین تخمین زده شده بود. پیش‌بینی گاموف، یک تابش گرمایی و به عبارتی تابش جسم سیاه بود. 
\cite{gamow1948evolution} 
تعداد فوتون‌های تابش زمینه حدود $10^{10}$ برابر تعداد باریون‌هاست. چگالی تعداد فوتون‌‌ها نیز ۴۱۱ بر هر $cm^{3}$ است.
\cite{dodelson2003modern}
 چگالی انرژی کیهان در
 $z > 4000$
 یا در دمای معادل
  $T > 10^4 K$
 با سهم فوتون‌ها غالب بوده است. 
\cite{durrer2015cosmic}

\begin{figure}
	\begin{center}
		\includegraphics[scale=0.3]{figs/dipole.png}
	\end{center}
	\caption[
	دوقطبی CMB در مختصات کهکشانی، مشاهده شده توسط ماهواره WMAP. خط قرمز در استوا اثر تابش کهکشان راه شیری است. این دوقطبی با کم کردن بهترین برازش تک‌قطبی از آسمان CMB و با حذف نوسانات سالانه دامنه حرکت زمین (سرعت میانگین $30 km/s $ )به دست آمده است.
	تصویر از صفحه WMAP.   
	]	
	{ دوقطبی CMB در مختصات کهکشانی، مشاهده شده توسط ماهواره WMAP. خط قرمز در استوا اثر تابش کهکشان راه شیری است. این دوقطبی با کم کردن بهترین برازش تک‌قطبی از آسمان CMB و با حذف نوسانات سالانه دامنه حرکت زمین (سرعت میانگین $30 km/s $ )به دست آمده است.
		تصویر از صفحه WMAP.   
		\footnotemark
	}
	\label{fig:dipole}
\end{figure}
\LTRfootnotetext{\url{http://map.gsfc.nasa.gov/mission/observatory cal.html}}

بلافاصله بعد از کشف تابش زمینه کیهانی، دانشمندان به پیدا کردن اثرات ناهمسانگردی در آن علاقه‌مند شدند. اگر ساختارهای کیهانی مانند کهکشان‌ها، خوشه‌ها، تهی‌جاها و فیلامان‌ها با فرض وجود اختلالات کوچک اولیه به وجود آمده اند، پس  چرا نباید این اختلالات بر روی CMB اثر بگذارند؟ برای زمانی طولانی تلاش برای پیدا کردن ناهمسانگردی‌ها بر روی تابش زمینه تنها منجر به مشاهده یک دوقطبی شد که در سال  ۱۹۶۹ توسط ماهواره WMAP
\LTRfootnote{Wilkinson Microwave Anisotropy Probe}
 گزارش شد.
\cite{conklin1969velocity}  
این دوقطبی در شکل 
\ref{fig:dipole}
نشان داده شده است. در سال ۱۹۸۹ ماهواره  CoBE
\LTRfootnote{Cosmic Background Explorer}
نه تنها طیف دقیقی از تابش زمینه به دست آورد، بلکه توانست افت وخیزهایی از مرتبه $10^{-5}$ را آشکارسازی کند. افت و خیزها را در هر نقطه از آسمان اینگونه تعریف می‌کنیم:
\begin{equation}
\frac{\delta T}{T}(\theta , \phi) = \frac{T(\theta , \phi) - \langle T \rangle}{\langle T \rangle}
\end{equation} 

طیف به دست آمده از CoBE با دقت بی‌نظیری مطابق با طیف جسم سیاه است و به عبارتی تابش زمینه کامل‌ترین نمونه یک تابش جسم سیاه شناخته شده در طبیعت است.  
شدت یک گاز فوتونی که طیف جسم سیاه دارد به شکل رابطه زیر است:
\cite{dodelson2003modern}
\begin{equation}
I_{\nu} = \frac{4 \pi \hbar \nu^3 /c^2}{\exp{(2\pi \hbar \nu /K_B T)} + 1}
\end{equation} 

نمودار طیف تابش زمینه کیهانی در شکل 
\ref{fig:Bbody}
آمده است.
\begin{figure}[h!]
	\begin{center}
		\includegraphics[width=0.6\textwidth]{figs/Bbody.png}
	\end{center}
	\caption[]	
	{    
		طیف تابش زمینه کیهانی و مطابقت بی نظیر با تابش جسم سیاه. این طیف مطابق با طیف تابش جسم سیاه با دمای 
		$T = 2.728 K$
		است. داده‌ها از 
		\cite{kogut2006arcade}
		و شکل از منبع
		\cite{durrer2008cosmic}
	}
	\label{fig:Bbody}
\end{figure}
تابع توزیع فوتون‌های CMB نیز مطابق با توزیع بوزونی است که به شکل 
\begin{equation}
f(E) = \frac{1}{e^{E/K_B T} -1 }
\end{equation} 
است و در طول انبساط کیهان توزیع خود را حفظ می‌کند.
برای اینکه منشا تابش زمینه را متوجه شویم نیاز داریم فرآیندی که در آن ماده باریونی از پلاسمای یونیزه بودن به اتم خنثی تبدیل می‌شود را مطالعه کنیم. برای توضیح این بخش از منبع 
\cite{ryden2017introduction}
استفاده می‌کنیم. تولید اتم خنثی ارتباط عمیقی با شفاف شدن کیهان از پس یک دوره کدر دارد. برای متوجه شدن این فرآیند نیاز داریم که سه دوران خاص در کیهان را به تفکیک بشناسیم.
\\
اولین دوره مورد نظر ما زمان «بازترکیب»
\LTRfootnote{Recombination}
است که به زمانی اشاره دارد که مولفه باریونی کیهان از یونیزه بودن به حالت خنثی تبدیل می‌شود. 
زمان دوم مورد نظر ما زمان «واجفتیدگی فوتون‌ها»
\LTRfootnote{photon decoupling}
که زمانی است که نرخ پراکندگی فوتون‌ از الکترون کمتر از نرخ انبساط کیهان می‌شود. وقتی فوتون‌‌ها از برهم‌کنش با الکترون‌‌ها دست برمیدارند عالم شفاف می‌شود.
دوره سوم نیز «سطح آخرین پراکندگی»
\LTRfootnote{Last scattering surface}
است که ما پشت این سطح را نمی‌بینیم چون قبل آن کیهان کدر است. سطح آخرین پراکندگی سطحی است که فوتون دست‌خوش آخرین پراکندگی از ماده می‌شود. اطراف هر ناظر در جهان یک سطح آخرین پراکندگی وجود دارد که فوتون‌های  CMB در آن آزادانه جریان دارند و از الکترون‌ها پراکنده نمی‌شوند. وقتی که نرخ انبساط کیهان از نرخ پراکندگی الکترون و پروتون سریع‌تر باشد احتمال اینکه فوتون از الکترونی پراکنده شود کم است لذا آخرین پراکندگی خیلی نزدیک به دوران واجفتیدگی فوتون‌ها است.
\begin{figure}[h!]
	\begin{center}
		\includegraphics[scale=0.35]{figs/lss.png}
	\end{center}
	\caption	
	{ هر ناظر یک سطح آخرین پراکندگی در اطراف خود دارد که بعد از آن سطح کیهان برایش شفاف شده است و قبل از آن کدر بوده است. فوتون‌ها بعد از آخرین پراکندگی بدون برهم‌کنش با ماده آزادانه حرکت می‌کنند اما در اثر انساط کیهان دچار قرمزگرایی و بلندشدن طول موج می‌شوند. شکل از منبع
		\cite{ryden2017introduction}
	}
	\label{fig:lss}
\end{figure}
 شکل
\ref{fig:lss}
به شهود بهتر درباره سطح آخرین پراکندگی کمک می‌کند.   
 در کیهان اولیه زمانی که ذرات در حال برهم‌کنش بودند، الکترون و پروتون با هم ترکیب می‌شدند و اتم هیدروژن خنثی می‌ساختند.اما به دلیل حضور فوتون‌های بسیار پرانرژی اتم هیدروژن ساخته شده به سرعت تجزیه می‌شد و به عبارتی واکنش زیر به صورت رفت و برگشتی در حال انجام بود.
\begin{equation}
p^{+} + e^{-} \rightleftharpoons H + \gamma
\end{equation} 
در اثر انبساط کیهان و پایین آمدن دما انرژي فوتون‌ها نیز کم می‌شود. کم شدن انرژی فوتون‌ها به الکترون و پروتون اجازه تشکیل اتم خنثی را می‌دهد. به عبارتی واکنش به سمت راست متمایل می‌شود. تشکیل اتم هیدروژن خنثی در دمای حدود 
$0.3 eV$ 
رخ می‌دهد. شاید در ابتدا انتظار داشته باشیم که این اتفاق در حدود انرژی بستگی اتم هیدروژن که مقدار 
$13.7 eV$
است رخ دهد اما این اتفاق نمی‌افتد. درست است که تعداد فوتون‌های پرانرژی به نسبت فوتون‌های کم‌انرژی‌تر کمتر است ولی از آن جایی که تعداد فوتون‌ها در کل خیلی بیشتر از باریون‌هاست، درصد کمی از فوتون‌ها پرانرژی هم برای مزاحمت ایجاد کردن برای تشکیل اتم خنثی کفایت می‌کند. تولید اتم هیدروژن خنثی از ذرات یونیزه یا همان بازترکیب در 
$ z \sim 1100$ 
که معادل دمای 
$T \sim 0.25 eV$
است رخ می‌دهد.
\cite{dodelson2020modern} 
بازترکیب عاملی است که باعث می‌شود واجفتیدگی فوتون‌ها اتفاق بیفتد. گرچه بدون وجود بازترکیب نیز واجفتیدگی اتفاق می‌افتاد، اما در زمان دیرتر تقریبا در 
$z = 43 $
. 
\cite{dodelson2003modern} 

\section{ ویژگی‌های تابش زمینه کیهانی}
\label{sec:cmb_physics}
در بخش قبل درباره برخی ویژگی‌های تابش زمینه کیهانی سخن گفتیم. اما درباره یک نکته صحبت نکردیم. ناهمسانگردی‌هایی که ماهواره CoBE مشاهده کرده است به چه دلیل ایجاد شده‌اند؟ و فوتون‌هایی که آزادانه در کیهان در حال حرکت اند، تحت تاثیر چه عواملی از بیرون قرار می‌گیرند؟
اثر دوقطبی که در تابش زمینه مشاهده شد از این حقیقت ناشی می‌شود که کیهان در زمان کنونی کاملا همگن نیست. از آن‌جایی که ما به دلیل گرانش توده‌های ماده اطرافمان حرکت شتاب‌دار داریم، شاهد انتقال دوپلری در CMB هستیم. افت و خیزها در مقیاس‌های کوچک زاویه‌ای که حتی با وجود حذف اثر دوقطبی توسط CoBE مشاهده شد نشان می‌دهد که کیهان در لحظه آخرین پراکندگی نیز کاملا همگن نبوده است. نوبل فیزیک ۲۰۰۶ برای مشاهده ناهمسانگردی‌های تابش زمینه اهدا شد. منشا بخشی از این ناهمسانگردی‌ها به کیهان اولیه برمی‌گردد. مدل‌های تورمی با فرض وجود یک دوره انبساط شتاب‌دار تندشونده در آغاز کیهان پاسخ بسیاری از سوالات ما در کیهان‌شناسی خواهد بود. تورم علاوه بر اینکه می‌تواند مشکل تختی، افق و تک‌قطبی را حل کند، قادر به توضیح ناهمسانگردی‌های اولیه تابش زمینه نیز خواهد بود. تورم بذر ایجاد اختلالات می‌شود. در پایان تورم همه نقاط در زمان کاملا یکسانی از انبساط شتاب مثبت نمی‌ایستند و اختلال‌های کوچک در چگالی ناشی از افت وخیزهای کوانتومی به وجود می‌آورند. آثار این افت و خیزها که از افق خارج می‌شوند در تابش زمینه مشاهده می‌شود. این ناهمسانگردی‌ها به صورت گاوسی توزیع می‌شوند. اما انحراف از گاوسی بودن در ناهمسانگردی‌های CMB در بعضی نظریات پیش‌بینی می‌شود. اگر روزی بتوانیم ردّ ناگوسی بودن را در تابش زمینه پیدا کنیم به رد و یا تایید نظریات کمک شایان می‌کند. یکی از عواملی که باعث ایجاد ناهمسانگردی‌های ناگوسی می‌شود ریسمان‌های کیهانی هستند که در فصل بعد آن را بررسی می‌کنیم. 
\subsection{اثر سَکس-وُلف}
هنگامی که فوتون‌های CMB در زمان واجفتیدگی از باریون‌ها جدا می‌شوند، فوتون‌ها که بر روی ژئودزیک‌ها حرکت می‌کنند در مسیر خود تحت تاثیر پتانسیل‌های گرانشی قرار می‌گیرند. اگر این پتانسل در طول زمان تغییر کند، فوتونی که در پتانسیل گرانشی سقوط می‌کند برای فرار از آن باید از پتانسیل عمیق‌تر/کم‌عمق‌تر عبور کند که منجر به انتقال به قرمز/ انتقال به آبی در انرژی فوتون می‌شود.
\cite{durrer2015cosmic}
این پدیده به اثر سکس-ولف 
\LTRfootnote{Sachs-Wolfe}
مشهور است.
\cite{sachs1967perturbations} 
اختلاف پتانسیل گرانشی در سطح آخرین پراکندگی یا lss با پتانسیل گرانشی در زمان کنونی مساوی با اختلاف بین افت و خیز‌های دما در این دو زمان است. یعنی: 
\begin{equation}
	\left(\frac{\Delta T}{T} \right)_O - \left(\frac{\Delta T}{T} \right)_{LSS} = \Phi_O-\Phi_{LSS}
\end{equation}
پتانسیل $\Phi_O$  تاثیری در ناهمسانگردی‌های تابش زمینه ندارد لذا از آن صرف‌نظر می‌کنیم. دما با عکس عامل مقیاس رابطه دارد، یعنی:
\begin{equation}
T \sim \frac{1}{a} \rightarrow  \frac{\Delta T}{T} \sim - \frac{\delta a}{a} 
\end{equation}
از معادله حالت می‌دانیم که $p=w\rho$. پس
\begin{align}
	a \propto t^{\frac{2}{3(1+w)}} \rightarrow \frac{\delta a}{a}=\frac{2}{3(1+w)}\frac{\delta t}{t}\\
	\left(\frac{\Delta T}{T} \right)_{LSS}=-\frac{2}{3(1+w)}\frac{\delta t}{t}
\end{align} 
\begin{equation}
	\left(\frac{\Delta T}{T} \right)_{LSS}=-\frac{2}{3} \frac{\delta t}{t}=\frac{2}{3}\Phi
\end{equation}
آخرین پراکندگی در زمان ماده-غالب رخ می‌دهد یعنی $w=0$. پس داریم
\begin{align}
	\frac{\Delta T}{T}(\hat{n})=\left(\frac{\Delta T}{T} \right)_O=-\Phi_{LSS}+\left(\frac{\Delta T}{T} \right)_{LSS} 
	=-\Phi_{LSS}+\frac{2}{3}\Phi_{LSS}=-\frac{1}{3}\Phi_{LSS}
\end{align}
بنابراین ارتباط میان افت‌وخیز دمایی به پتانسیل گرانشی سطحِ آخرین پراکندگی به این شکل است:
\begin{equation}
	\frac{\Delta T}{T}(\hat{n})=-\frac{1}{3}\Phi_{LSS}
\end{equation}
اگر انتگرال پتانسیل‌های سرراهی را نیز وارد کنیم، به اثر سَکس-ولف تجمعی
\LTRfootnote{Integrated Sachs-Wolfe(ISW) }
می‌رسیم.
\cite{alireza}
\begin{equation}
	\frac{\Delta T}{T}(\hat{n}) = - \frac{1}{3} \Phi_{LSS} ( \tau_{LSS} , \hat{n}) - 2 \int_{\tau_{LSS}}^{\tau_0} d \tau \Phi^\prime(\tau, \mathbf{x})
\end{equation}
   
\subsection{ طیف توان}
طیف توان زاویه‌ای
\LTRfootnote{angular power spectrum}
با $C_l$ داده می‌شود که حاوی اطلاعات بسیار ارزشمندی درباره تابش زمینه کیهان است. ناهمسانگردی‌های CMB توزیعی گاوسی دارند که در نتیجه آن تمام خواص تابش با این توزیع از ممان دوم آن به دست می‌آید. در فضای حقیقی تابع هم‌بستگی دو نقطه‌ای تعریف کنیم که در فضای فوریه تبدیل به $C_l$ یا طیف توان می‌شود. طیف توان به دست آمده برای داده‌های تابش زمینه پلانک ۲۰۱۸ در شکل
\ref{fig:power}
آمده است.
\par
در این فصل با تابش زمینه کیهان به عنوان فوتون‌های بازمانده از مهبانگ آشنا شدیم و دانستیم که مطالعه این تابش می‌تواند اطلاعات بی‌نظیری از کیهان اولیه به ارمغان بیاورد. در فصل بعد نیز با موجودات یک بعدی به نام ریسمان‌های کیهانی آشنا خواهیم شد و ردّپاهایی که از خود بر تابش زمینه باقی خواهند گذاشت را بررسی خواهیم کرد. 
\begin{figure}
	\begin{center}
		\includegraphics[scale=0.35]{figs/powerspec.png}
	\end{center}
	\caption	
	{	طیف توان داده‌های دمایی تابش زمینه کیهان به دست آمده از داده‌های ماهواره پلانک ۲۰۱۸.
		\cite{akrami2018legacy}
	}
	\label{fig:power}
\end{figure}


