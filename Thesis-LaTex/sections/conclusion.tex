همانطور که در فصل 
\ref{ch:cmb}
درباره آن صحبت کردیم، دمای تابش زمینه کیهانی افت‌و‌خیز‌هایی از مرتبه $10^{-5}$ دارد. ناهمسانگردی‌های CMB منشا متفاوتی دارند. بخشی از این ناهمسانگردی‌ها از کیهان اولیه می‌آیند. اما عوامل دیگری نیز می‌توانند در ایجاد ناهمسانگردی در CMB تاثیرگذار باشند. در فصل
\ref{ch:cosmic_string}  
گفتیم که ریسمان‌های کیهانی، نواقص توپولوژیکی هستند که وجود آن‌ها در برخی نظریات تورمی پیش‌بینی شده است و ممکن است در کیهان اولیه و در اثر شکست تقارن ناشی از پایین آمدن دما تشکیل شده باشند. ریسمان‌های کیهانی یکی از عواملی هستند که می‌توانند ناهمسانگردی در تابش زمینه ایجاد کنند. کمیت مربوط به شدت ریسمان که معرفی کردیم، کمیت بی‌بعد تنش ریسمان یا $G\mu$ است. (که $\mu$ چگالی بر واحد خط ریسمان و G ثابت گرانش است.) مشاهده ریسمان می‌تواند دریچه جدیدی به سوی کیهان اولیه باشد و رد و تایید نظریه‌های موجود مرتبط کمک کند. قید گذاشتن بر مقدار $G\mu$ نیز به جهت محدود کردن بازه جستجو مفید است. 
تا کنون تحقیقاتی در زمینه مقید کردن مقدار $G\mu$ انجام گرفته است که هر کدام با توجه به فرضیاتی که در حل مسئله به کار برده اند، حد بالایی را برای تنش ریسمان پیشنهاد کرده‌اند. به طور مثال بررسی طیف تابش زمینه برای داده‌های ماهواره پلانک مقادیر 
$f_{10} \leq 0.015 $
که معادل 
$G\mu \leq 1.5 \times 10^{-7}$
برای مدل ریسمان نامبو-گاتو و 
$f_{10} \leq 0.033 $
یا 
$G\mu \leq 3.6 \times 10^{-7}$
را برای مدل آبلین-هیگز گزارش کرده است.
\cite{ade2014planck}
با استفاده از پردازش تصویر و مطالعات آماری مقدار  
$G\mu \geq 1.2 \times 10^{-7}$
برای تنش ریسمان
 \cite{vafaei2017multiscale}
و با استفاده از روش‌های یادگیری ماشین درختی مقدار 
$G\mu \geq 3.0 \times 10^{-8}$
برای شبیه‌سازی‌های نسل چهارم تابش زمینه گزارش شده است. 
\cite{vafaei2018cosmic}
استفاده از روش شبکه‌های عصبی نیز مقدار
$G\mu \geq 2.3 \times 10^{-9}$
برای شبیه‌سازی بدون نوفه
\cite{ciuca2017bayesian, ciuca2019inferring}
 و 
 $G\mu \geq 1.2 \times 10^{-7}$
 برای داده با نوفه شبه تلسکوپ آتاکاما گزارش شده است.
 \cite{ciuca2020information}
در این تحقیق برای ردیابی ریسمان از اثر گات-کایزر-استبینز بر تابش زمینه کیهانی استفاده کردیم. ابزاری که برای 
تشخیص تنش ریسمان در نقشه‌های تابش زمینه به کار بردیم، شبکه‌های عصبی پیچشی است. در فصل
\ref{ch:deep_learning}
درباره علم داده و یادگیری ماشین سخن گفتیم و شبکه‌های عصبی را معرفی کردیم. گفتیم که مزیت روش‌های مبتنی بر یادگیری عمیق این است که ماشین با داده‌های خام آموزش می‌بیند و در حین فرآیند آموزش بهترین ویژگی‌هایی که در تصویر برای رسیدن به هدف مفید هستند را استخراج می‌کند. بنابراین ما را از چالش بزرگ انتخاب ویژگی‌های خوب و مناسب رها می‌کند. در این پژوهش قصد داشتیم به دو سوال پاسخ دهیم:
\\
کوچکترین تنش ریسمان قابل تشخیص در شبیه‌سازی‌های مختلف تابش زمینه چقدر است؟ \\
آیا می‌توانیم با استفاده از رهیافت یادگیری عمیق و شبکه‌های عصبی شبکه ریسمان را در داده‌های رصدی تابش زمینه تشخیص دهیم؟    \\
راهکار ما برای حل مسئله این است که نقشه شبیه‌سازی شده از تابش زمینه در حضور ریسمان را به عنوان ورودی و مقادیر  $G\mu$ متناظر با هر نقشه ورودی را به عنوان هدف و خروجی به یک شبکه عصبی پیچشی بدهیم تا ماشین مدلی بسازد که بتواند مقدار $G\mu$ را به هر نقشه ورودی متناظر کند. 
برای این کار در ابتدا مجموعه‌ای از شبیه‌سازی‌های متداول تابش زمینه را آماده کردیم تا با استفاده از شبیه‌سازی‌های شبکه ریسمان، نقشه‌ CMB در حضور ریسمان را بسازیم. برای نزدیک کردن شبیه‌سازی به رصدهای واقعی اثر بیم و نوفه رصدی را نیز بر نقشه‌ها اعمال کردیم. معرفی بخش‌های مختلف در شبیه‌سازی در فصل 
\ref{ch:simulations}
آورده شده است.پس از طی مراحل پیش‌پردازش، داده‌ها را به عنوان ورودی به ماشینی که بهینه‌ترین انتخاب باشد می‌دهیم تا آموزش ببیند. این فرآیند را برای سه دسته از آزمایش‌های رصدی تابش زمینه تکرار کردیم. دسته اول آزمایش شبه نسل چهار تابش زمینه، دسته دوم آزمایش شبه تلسکوپ کیهان‌شناسی آتاکاما و دسته سوم آزمایش‌های شبه ماهواره پلانک است. نتایج آموزش ماشین بر ۷ نوع از این آزمایش‌ها در جدول 
\ref{table:min-mes} 
و
\ref{table:min-det}
خلاصه شده است.
\par
پس از اینکه با استفاده از داده‌های شبیه‌سازی مدلی ساختیم که به هر نقشه ورودی بتواند مقدار $G\mu$ را متناظر کند، به سراغ پاسخ دادن به سوال دوم می‌رویم. ما می‌توانیم نقشه‌های تابش زمینه رصدی را به عنوان ورودی به ماشینی که از پیش با داده‌های شبیه‌سازی آموزش دیده است بدهیم تا مقدار تنش ریسمان موجود در داده‌ها را پیش‌بینی کند. ما از داده‌های CMB پلانک ۲۰۱۸ استفاده می‌کنیم که دقیق‌ترین رصد تابش زمینه هستند. سوالی که پیش می‌آید این است که از مدل  ساخته شده بر روی کدام یک از ۷ گونه آزمایشی که پیشتر اشاره کردیم استفاده کنیم. بدیهی است که باید به سراغ شبیه‌ترین آزمایش به داده‌های رصدی پلانک ۲۰۱۸ برویم. از بین آزمایش‌های شبه پلانک، نقشه‌های شبیه‌سازی End-to-End همانند داده‌های رصدی از الگوریتم تفکیک مولفه عبور کرده‌اند و لذا شبیه‌ترین آزمایش به رصد هستند. ماشینی که بر روی داده‌های آزمایش 
\lr{E2E}
آموزش دیده است، قادر است 
$G\mu \geq 8.8 \times 10^{-7}$
را اندازه‌گیری کند. با اعمال مدل \lr{E2E} بر داده‌های پلانک و مشاهده پیش‌بینی‌های ماشین برای آن، هیستوگرام پیش‌بینی‌های ماشین برای داده‌های CMB را رسم کردیم که در شکل
\ref{fig:hist_nsye2e}
آمده است. با توجه به اینکه پیش‌بینی‌های ماشین برای داده رصدی کاملا مشابه با توزیع پیش‌بینی‌های ماشین برای یک دسته از شبیه‌سازی‌های پوچ با تنش ریسمان صفر(و یا کمتر از حد یادگیری ماشین) در شکل
\ref{fig:nsye2e_cm}
است، نتیجه می‌گیریم که ماشین ریسمانی نمی‌بیند. پس مقدار  
$8.8 \times 10^{-7}$
را به عنوان حد بالای تنش ریسمان در داده‌های دمایی تابش زمینه کیهانی پلانک ۲۰۱۸ اعلام می‌کنیم.
\par
این تحقیق در ادامه دو مقاله
\cite{vafaei2017multiscale , vafaei2018cosmic}
انجام شده است. رویکرد تحقیق پیشین ، پردازش تصویر و استفاده از ویژگی‌های آماری برای دنبال کردن ردپای ریسمان بوده است. اما در این تحقیق از رویکرد یادگیری عمیق استفاده کردیم زیرا معتقدیم که شبکه‌های عصبی از قدرتمندترین ابزار برای آشکارسازی و استخراج ویژگی از تصاویر هستند که ما را از انتخاب میان بی‌شمار ابزار آماری و... بی‌نیاز می‌کنند. اعمال مدل بر روی داده‌های رصدی نیز رویکردی جدید است. تحقیق انجام‌شده طی این پایان‌نامه در دو کنفرانس داخلی ارائه شده است و پیش‌بینی می‌شود که با یک مقاله به پایان برسد.     

%\renewcommand{\arraystretch}{1.3}
%{
%\begin{table*}
%\begin{latin}
%\centering
%\begin{tabular}{ P{5cm} | P{4cm} | P{4cm}  }
%	\hline
%%	\multicolumn{3}{c}{} \\
%	\multicolumn{3}{c}{Minimum Measurable G$\mu$} \\
%
%	\hline
%	Experiment 			& Confusion Matrix 		& P value \\
%	\hline
%%	                    &						&			\\
%	CMB-S4-like(II) 	& 1.9 $\times10^{-7}$  	& 1.7 $\times10^{-7}$ \\
%	CMB-S4-like(I)  	& 1.9 $\times10^{-7}$	& 1.6 $\times10^{-7}$ \\
%	ACT-like 			& 4.1 $\times10^{-7}$	& 3.5 $\times10^{-7}$ \\
%	noise-free FFP10    & 8.6 $\times10^{-8}$	& 8.1 $\times10^{-8}$ \\
%	FFP10 				& 6.8 $\times10^{-7}$	& 3.6 $\times10^{-7}$ \\
%	noise-free E2E 		& 7.2 $\times10^{-7}$	& 1.9 $\times10^{-7}$   \\
%	E2E 				& 8.8 $\times10^{-7}$	& 4.9 $\times10^{-7}$ \\
%
%
%\end{tabular}
%\end{latin}
%\caption{حد کمینه قابل اندازه‌گیری برای انواع شبیه‌سازی‌های تابش زمینه استفاده شده در این تحقیق }
%\label{table:min-mes}
%\end{table*}
%}
%
%{
%		\renewcommand{\arraystretch}{1.3}
%	\begin{table*}
%		\begin{latin}
%			\centering
%			\begin{tabular}{ P{5cm} | P{4cm} | P{4cm}  }
%				\hline
%				%	\multicolumn{3}{c}{} \\
%				\multicolumn{3}{c}{Minimum Detectable G$\mu$} \\
%				
%				\hline
%				Experiment 			& Confusion Matrix 		& P value \\
%				\hline
%				%	                    &						&			\\
%				CMB-S4-like(II) 	& 8.9 $\times10^{-8}$  	& 2.6 $\times10^{-8}$ \\
%				CMB-S4-like(I)  	& 1.9 $\times10^{-7}$	& 1.3 $\times10^{-7}$ \\
%				ACT-like 			& 1.9 $\times10^{-7}$	& 2.4 $\times10^{-7}$ \\
%				noise-free FFP10    & 4.1 $\times10^{-8}$	& 3.2 $\times10^{-8}$ \\
%				FFP10 				& 4.1 $\times10^{-7}$	& 1.4 $\times10^{-7}$ \\
%				noise-free E2E 		& 4.1 $\times10^{-7}$	& 1.4 $\times10^{-7}$   \\
%				E2E 				& 8.9 $\times10^{-7}$	& 3.7 $\times10^{-7}$ \\
%				
%				
%			\end{tabular}
%		\end{latin}
%		\caption{حد کمینه قابل آشکارسازی برای انواع شبیه‌سازی‌های تابش زمینه استفاده شده در این تحقیق }
%		\label{table:min-det}
%	\end{table*}
}

