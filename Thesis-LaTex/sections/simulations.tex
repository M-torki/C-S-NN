
در فصل‌های
\ref{ch:cmb}
و
\ref{ch:cosmic_string}
ریسمان‌ها و ردپایشان را بر روی تابش زمینه کیهانی  شناختیم و در فصل 
\ref{ch:deep_learning}
شبکه‌های عصبی پیچشی را به عنوان ابزار مورد نظر در این تحقیق برای آشکارسازی ریسمان‌ها معرفی کردیم، در یک قدمی رسیدن به اصل داستان آشکارسازی قرار گرفته‌ایم. ما از علم داده و یادگیری ماشین صحبت کرده‌ایم پس پیش از هر چیز باید داده‌های مورد نیاز در این تحقیق را بشناسیم و بدانیم از چه داده‌ای سخن می‌گوییم. هدف نهایی ما بررسی ردپای ریسمان بر داده‌های رصدی تابش زمینه است. برای این کار از دقیق‌ترین داده‌های رصدی حال حاضر، یعنی داده‌های دمایی تابش زمینه، رصد شده توسط ماهواره پلانک
\LTRfootnote{Planck} 
 مربوط به سال ۲۰۱۸ استفاده می‌کنیم. پس در ابتدای این فصل داده‌های رصدی CMB را معرفی می‌کنیم. در ادامه باید شبیه‌سازی‌هایی که مورد استفاده ما در این تحقیق است را نیز معرفی کنیم.
 \par   
 ممکن است این سوال پیش بیاید که چرا از شبیه‌سازی استفاده می‌کنیم؟ مگر نه این است که داده‌های حقیقی را داریم؟ برای پاسخ به این سوال به مسئله خودمان رجوع می‌کنیم. در یادگیری عمیق و شبکه‌های عصبی با مدل‌هایی سر و کار داریم که از حدود چند هزار یا حتی چند میلیون پارامتر آزاد دارند. برای آموزش چنین شبکه‌هایی بالطبع نیاز به حجم عظیمی از داده داریم. این در حالی است که ما تنها یک کیهان و یک تابش زمینه داریم. داده‌های رصدی تابش زمینه هرگز برای هدف ما کافی نخواهد بود. پس به سراغ شبیه‌سازی می‌رویم. البته دلایل دیگری نیز وجود دارند که استفاده از شبیه‌سازی‌ها را لازم می‌کند که خارج از بحث ماست. \\
 نقشه‌های ورودی ماشین برای آموزش باید تا حد ممکن به داده‌های رصدی شبیه باشند. نقشه‌های مورد نظر ما ترکیبی از شبیه‌سازی‌های CMB گاوسی و شبیه‌سازی ریسمان‌های کیهانی است. هم‌چنین برای ساخت یک نقشه کامل که قابل مقایسه با داده‌های رصدی باشد، اثر بیم و هم‌چنین اثرات نوفه رصدی را نیز باید در نظر گرفت که در ادامه فصل جزئیات بخش‌های مختلف نامبرده شده شرح داده شده است.  
 \section{داده‌های رصدی تابش زمینه } 
کیهان‌شناسی امروز به عنوان علمی دقیق و محاسباتی را مدیون پروژه‌های رصدی هستیم که داده‌های بزرگ و با دقت از آسمان جمع‌آوری می‌کنند. در زمینه رصد تابش زمینه کیهانی در ابتدا ماهواره CoBE در سال ۱۹۸۹ شروع به کار کرد که توانست یک دوقطبی در تابش زمینه مشاهده کند. از آن‌جایی که منشا این دوقطبی انتقال دوپلری ناشی از حرکت ناظر نسبت به آخرین سطح پراکندگی است و منشا کیهانی ندارد به بررسی این پدیده علاقه‌ای نداریم. ماهواره WMAP در سال ۲۰۰۱ توسط NASA  
\LTRfootnote{National Aeronautics and Space Administration}
به فضا فرستاده شد و به مدت ۹ سال داده‌گیری از آسمان انجام داد. داده‌گیری WMAP در ۵ باند فرکانسی از 
$22.8GHz$ 
تا
$93.5GHz$
انجام شده است که در تمام آن‌ها به قطبش نیز حساس هستند. نتایج این پروژه رصدی در منبع 
\cite{hinshaw2013nine}
آمده است و داده‌های آن نیز در دسترس عموم قرار دارد.
جدیدترین پرتاب ماهواره رصدی برای مطالعه تابش زمینه کیهانی که توسط ESA
\LTRfootnote{European Space Agency}
انجام گرفته است، ماموریت پلانک
\LTRfootnote{\url{http://www.rssd.esa.int/planck}}
است. پلانک به منظور جمع‌آوری اطلاعات CMB از گیرنده‌های رادیویی بسیار حساس در دماهای بسیار پایین استفاده می‌کند. این گیرنده‌ها دمای معادل جسم سیاه تابش زمینه را معین می‌کنند و قادر به تشخیص تغییرات دما از حدود $\mu K$ هستند.
بلندای فضاپیمای پلانک $4.2 m $ و قطر بیشنه آن $4.2 m $ است و ۱.۹ تن جرم دارد. ماهواره پلانک در سال ۲۰۰۹ پرتاب شد و برنامه‌ریزی شده بود که تا ۳ سال داده‌گیری انجام دهد. این ماهواره در نوامبر ۲۰۱۳ به کار خود پایان داد. پلانک دارای یک آرایه از ۷۴ آشکارساز است که در بازه فرکانسی   $25 GHz$ تا  $1000 GHz$ ، شامل ۳ باند فرکانسی پایین (LFI)
\LTRfootnote{Low Frequency Instrument}
و ۶ باند فرکانسی بالا (HFI)
\LTRfootnote{High Frequency Instrument}
است.دقت و توان آشکارسازی در هر کدام از باندهای فرکانسی متفاوت است. پلانک در هرسال دوبار از کل آسمان تصویربرداری می‌کند و تفکیک زاویه‌ای آن در فرکانس‌های مختلف، متفاوت و بین ۵ تا ۳۵ دقیقه متغیر است.
\cite{akrami2018legacy} 
نقشه کلی به دست آمده از پلانک برای تاش زمینه، برآیندی از داده‌های مربوط به فرکانس‌های مختلف است و هر فرکانس وزن مربوط به خود را دارد. 
\par
علاوه بر ناهمسانگردی‌های اولیه که هدف اصلی بررسی در ماموریت پلانک است، مولفه‌های اخترفیزیک دیگری نیز که در وابستگی‌شان به فرکانس و ویژگی‌های مکانی با یکدیگر تفاوت دارند توسط آشکارساز‌ها دریافت می‌شود. این مولفه‌ها به اثرات‌ «پیش‌زمینه»
\LTRfootnote{Foreground}
 معروف است(در برابر آن کلمه «زمینه» را داریم که مربوط به تابش CMB می‌شود و نباید زمینه و پیش‌زمینه را باهم قاطی کنیم!:) ) با انجام روش‌هایی می‌توان در فرکانس‌های مختلف پیش‌زمینه را مشخص  و ناهمسانگردی‌های اولیه را از آلودگی وجود آن‌ها تا حد بی‌مانندی پاک و تمیز کنیم. برای پاک کردن نقشه‌های CMB از اثرات پیش‌زمینه ۴ روش مختلف وجود دارد.  این چهار روش که از ۴ الگوریتم مختلف برای این کار استفاده می‌کنند، عبارتند از:
  \begin{itemize}
  	\item Commander
  		\cite{eriksen2008joint , adam2016planck} 
  	\item NILC  
  	  	\cite{basak2013needlet} 
	\item SEVEM 
  	    \cite{leach2008component}  	
	\item SMICA 
  	    \cite{collaborationplanck} 	
  \end{itemize}
تفاوت‌های بین نقشه‌های به دست آمده از این ۴ روش را می‌توان به عنوان یک تخمین از عدم قطعیت در بازسازی CMB قلمداد کرد که این مقدار به طور اطمینان‌بخشی کوچک است.
\cite{akrami2018planck}
لذا در این تحقیق ما به استفاده از یکی از این روش‌ها بسنده می‌کنیم. انتخاب ما نقشه SMICA از داده‌های پلانک ۲۰۱۸ است.

\begin{figure}
	\begin{center}
		\includegraphics[scale=0.1]{figs/9frq.png}
	\end{center}
	\caption{ 
		افت و خیزهای رصدی آسمان دمایی تابش زمینه در ۹ باند فرکانس رصدشده توسط پلانک. مولفه معمول دوقطبی در اینجا حذف شده است. 
		\cite{akrami2018legacy}
	}
	\label{fig:9frq}
\end{figure}

\section{شبیه‌سازی‌ها}
برای ساخت شبیه‌سازی‌هایی از تابش زمینه که ردّپای ریسمان در خود دارند نیاز داریم که خودمان اثر ریسمان را به شبیه‌سازی‌های CMB بدون ریسمان اضافه کنیم. ناهمسانگردی‌های ناشی از وجود ریسمان، ناگاوسی هستند در حالی که ناهمسانگردی‌های موجود در شبیه‌سازی‌های معمول تابش زمینه توزیعی گاوسی دارند. در این بخش سهم‌های مختلفی که باید برای ساخت یک شبیه‌سازی مطلوب در نظر گرفت را معرفی می‌کنیم.
\subsection{شبیه‌سازی‌های گاوسی تابش زمینه}
\label{sec:cmb_sims}
% \underline{\textbf{شبیه‌سازی‌های گاوسی تابش زمینه}}:
 در قسمت گاوسی شبیه‌سازی نقشه کامل افت و خیزها از یک تابع توزیع گاوسی تبعیت می‌کند. برای شبیه‌سازی این افت و خیزها به مدل استاندارد کیهان‌شناسی و مدل تورمی رجوع می‌کنیم. در این پژوهش ما سه دسته آزمایش رصدی را مورد تحقیق قرار داده‌ایم و برای شبیه‌‌سازی‌هایمان از آن‌ها استفاده کرده‌ایم. این آزمایش‌ها به شرح زیر است. 
 \begin{itemize}
 	\item آزمایش شبه تلسکوپ آتاکاما
 	\cite{aiola2020atacama}
 	
	 \item آزمایش شبه نسل چهارم تابش زمینه 
 	\LTRfootnote{CMB-Stage4}
 	\cite{abazajian2016cmb}
 	
    \item آزمایش شبه پلانک 
 \end{itemize}
 دسته اول شبیه به به رصدهای تلسکوپ کیهان‌شناسی آتاکاما 
 \LTRfootnote{Atacama Cosmology Telescope (ACT)}
  دسته دوم  رصدهای نسل چهارم تابش زمینه	و دسته سوم رصدهای ماهواره پلانک ۲۰۱۸ است. تلسکوپ آتاکاما یک تلسکوپ ۶ متری در صحرای آتاکاماست که در فرکانس‌های ۱۴۵، ۲۱۵ و ۲۸۰ گیگاهرتز داده‌های تابش زمینه را دریافت می‌کند و نسل چهارم تابش زمینه نیز نسل بعدی رصدهای CMB است که در آینده قرار است داده‌گیری انجام دهد.  
متناظر با هر کدام از این رصدها از شبیه‌سازی‌هایی استفاده کردیم که مطابق با ویژگی‌های رصدی این سه دسته بوده‌اند. در بخش‌های آتی این رصدها و شبیه‌سازی‌های متناظرشان را به اختصار توضیح می‌دهیم.
آزمایش‌های شبه تلسکوپ آتاکاما و هم‌چنین آزمایش‌های نوع اول و دوم نسل چهارم تابش زمینه با استفاده از $HEALPix$
 \LTRfootnote{Hierarchical Equal Area isoLatitude Pixelization}
 \LTRfootnote{\url{healpix.sourceforge.io}}
 این آزمایش‌ها را شبیه‌سازی کرده‌ایم. این دسته شبیه‌سازی‌ها با استفاده از طیف توان CMB مطابق با داده‌های رصدی ۲۰۱۸ به دست می‌آیند. 
 در جدول 
\ref{Table:com_sim}
 به ویژگی‌های منحصر به فرد هر گونه از شبیه‌سازی‌ها اشاره شده است.
% \subsection{رصدهای نسل چهارم تاش زمینه}
% \subsection{تلسکوپ کیهان‌شناسی آتاکاما}
 
  \begin{figure}
 	\begin{center}
 		\includegraphics[scale=0.5]{figs/healpixs.png}
 	\end{center}
 	\caption{ 
 		شبیه‌سازی‌های
 		\lr{Healpix}
 		، از سمت چپ به راست
 		\lr{CMB-S4 II} ، \lr{CMB-S4 I}  و \lr{ACT}
 		. برای اینکه تفاوت شبیه‌سازی‌ها واضح‌تر باشد، تصاویر با فیلتر شار پردازش شده اند. 
 	}
 	\label{fig:healpix}
 \end{figure}

% \subsection{ماهواره پلانک}
 دسته سوم از این آزمایش‌ها مشابه داده‌های CMB پلانک ۲۰۱۸ است. این دسته شامل شبیه‌سازی 
 \lr{FFP10}
 \LTRfootnote{Full Focal Plane}
   و نقشه‌های End-to-End یا 
   \lr{E2E}
   است که توسط گروه پلانک و طبق جدیدترین داده‌های رصدی و پارامترهای کیهان‌شناسی مطابق با مشاهدات پلانک ۲۰۱۸ ساخته شده است.
مانند رصدهای ماهواره پلانک در ۹ فرکانس مختلف برای از ۳۰ تا ۸۷۵ گیگاهرتز و برای هر فرکانس رصدی ۱۰۰۰ نمونه نقشه کامل آسمان برای شبیه‌سازی 
  \lr{FFP10}
 خالص توسط پلانک عرضه شده است.
 \cite{ffp}
 \\ همانطور که در بخش داده رصدی گفتیم، رصدهای پلانک در ۹ فرکانس انجام می‌شود اما نقشه نهایی برآیندی از همه فرکانس‌هاست و یک مرحله نیز برای تفکیک مولفه پیش‌زمینه از زمینه در آن انجام شده است. نقشه‌های  \lr{E2E} نیز مانند داده‌های رصدی تفکیک مولفه شده پلانک، \lr{FFP10} هایی هستند که از الگوریتم تفکیک مولفه عبور کرده‌اند. 
از آنجایی که داده‌های رصدی نیز از این پروسه عبور کرده‌اند و الگوریتم مشابهی روی آن‌ها اعمال شده است، لذا این شبیه‌سازی‌های تفکیک مولفه شده قابل‌مقایسه‌ترین نقشه‌ها با داده رصدی هستند. شبیه‌سازی‌های تفکیک مولفه‌ شده 
 ۹۹۹ نقشه آسمان کامل 
 \lr{E2E}
  به ازای هر روش تفکیک مولفه یعنی 
 \lr{SMICA}, \lr{NILC}, \lr{SEVEM}
 و 
 \lr{Commander}
 ارائه شده است.
 \cite{ade2016planck}
  همه این شبیه‌سازی ها در گنجینه پلانک
 :) در مسیر زیر دسترس اند:
 \begin{latin}
	\textbf{FFP10}: \\
pla.esac.esa.int $\to$ maps $\to$ simulations $\to$ cmb \\
	\textbf{E2E}: \\
pla.esac.esa.int $\to$ advanced\;search\;and\;map\;operations $\to$ simulated\;maps\;search $\to$ comp-separation
 \end{latin}
همانند داده‌های رصدی ما روش SMICA را برای تفکیک مولفه در نقشه‌های 
\lr{E2E}
استفاده می‌کنیم. 

 
\subsection{شبیه‌سازی اثر بیم تلسکوپ} 
% \\
% \par
% \underline{\textbf{شبیه‌سازی اثر بیم تلسکوپ}}:
 %\par
 زمانی که تلسکوپ به سمت یک ناحیه خاص نشانه می‌رود تنها از سمت آن ناحیه نور دریافت نمی‌کند و اطلاعاتی از محدوده‌های دیگر نیز به آن می‌رسد. این اثر که به دلیل محدود بودن دهانه تلسکوپ اتفاق می‌افتد اثر بیم
  \LTRfootnote{Beam}
  نام دارد و باعث تار و مات شدن تصویر دریافتی می‌شود. از آن جا که اثر ییم برخی جزییات را نابود می‌کند یافتن ردّ پاهای کوچک داخل تصویر را سخت می‌کند. برای اینکه شبیه‌سازی‌ها هر چه بیشتر به آنچه که در واقعیت توسط تلسکوپ رصد می‌شود شباهت داشته باشد باید اثر بیم را نیز به نقشه‌ها اعمال کنیم. 
 اعمال اثر بیم با استفاده از رهیافتی که در مرجع  
 \cite{Bond:1987ub}
 معرفی شده است با کمک یک فرآیند پیچش با تابع مشخصه رصدگر یعنی $\mathcal{B}(\vec{k}-\vec{k}';\Gamma)$ انجام می‌شود. یعنی 
 \begin{equation} 
 \label{convlo} 
 {\mathcal{F}}_{B}(\vec{k},\Gamma)=\int d\vec{k}'\mathcal{B}(\vec{k}-\vec{k}';\Gamma){\mathcal{F}}(\vec{k}'-\vec{k}) 
 \end{equation} 
 که در معادله بالا $\Gamma={\rm FWHM}/\sqrt{8\ln 2}$ و ${\rm FWHM}$ پهنای تابع بیم در نصف بیشینه 
\LTRfootnote{Full-Width at Half Maximum}
 است. هم‌چنین 
 $\mathcal{F}(\textbf{r})\equiv [T(\textbf{r})-\langle T(\textbf{r})\rangle]/\langle T(\textbf{r})\rangle$
 است.
%\cite{alireza}
\subsection{شبیه‌سازی اثر نوفه} 
% \\
% \par
% \underline{\textbf{شبیه‌سازی اثر نوفه}}:
 برای شبیه‌سازی‌های 
 \lr{E2E}
 ۳۰۰ نقشه نوفه
 \LTRfootnote{Noise}
  متناظر با هر کدام از روش‌های تفکیک مولفه وجود دارد که بر روی گنجینه پلانک در مسیر زیر در دسترس است:\\
 \begin{latin}
  pla.esac.esa.int $\to$ advanced\;search\;and\;map\;operations $\to$ simulated\;maps\;search $\to$ comp-separation $\to$ noise	
 \end{latin}

   اما برای سایر شبیه‌سازی‌های رصدی ما از مدل نوفه سفید غیرهم‌بسته استفاده کرده‌ایم.
 میزان نوفه موجود در یک نقشه با معیار 
 \lr{SNR}
 \LTRfootnote{Signal-To-Noise Ratio}
 داده می‌شود که بیانگر نسبت انحراف معیار نقشه سیگنال به انحراف معیار نقشه نوفه است. این مقدار برای آزمایش‌های مختلف متفاوت است. مقدار \lr{SNR} برای آزمایش‌های شبه تلسکوپ آتاکاما، مانند شبیه‌سازی‌های پلانک ۱۰ است و برای آزمایش‌های شبه نسل چهارم تابش زمینه مقادیر ۱۵ و ۲۰. هم‌چنین مقدار \lr{SNR} در نقشه‌های نوفه شبیه‌سازی 
    \lr{E2E}
    ، ۸ است. جدول 
 \ref{Table:com_sim}
 به تفکیک ویژگی‌های متمایز کننده آزمایش‌های رصدی مختلف را مشخص می‌کند.
 %------------------------------------------------------------------------------------------------------------%

\begin{table}[H]
	\centering
	\caption{ویژگی‌های آزمایش‌های مختلف رصدی تابش زمینه استفاده شده در این تحقیق}
	\label{Table:com_sim}
	 \begin{latin}
	\begin{tabular}{cccc}
		\hline
		Experiment     &    $ N_{\rm side}$        & SNR             & FWHM   \\ \hline
		CMB-S4-like(II)    & 4096         & 20              & $0.9'$ \\
		CMB-S4-like(I)     & 4096         & 15              & $0.9'$ \\
		ACT-Like           & 4096         & 10              & $0.9'$ \\
		FFP10        & 2048         & 10              & $5'$   \\
		E2E     & 2048      & 8     & $5'$   \\ \hline
	\end{tabular}
 \end{latin}
\end{table}

 %------------------------------------------------------------------------------------------------------------%
\subsection{شبیه‌سازی اثر ریسمان‌های کیهانی} 
 \label{sec:string_sims}
% \par
% \underline{\textbf{شبیه‌سازی اثر ریسمان‌های کیهانی}}:
منظور ما از شبیه‌سازی اثر ریسمان، مدل کردن ناهمسانگردی‌های غیرگاوسی که ریسمان‌های کیهانی بر تابش زمینه ایجاد می‌کنند است. برای شبیه‌سازی‌های ریسمان فرض‌های متفاوتی وجود دارد. به طور مثال دو نوع از شبیه‌سازی ریسمان نامبو-گاتو
\LTRfootnote{Nambu-Goto}
\cite{fraisse2008small}
و آبلین-هیگز
\LTRfootnote{Abelian-Higgs}
\cite{moore2001evolution , bevis2007cmb}
هستند که در کنش متفاوت اند. شبیه‌سازی شبکه ریسمانی که ما استفاده کرده‌ایم شبیه‌سازی نامبو-گاتو است که از روی اثر سکس-ولف تجمعی تابش زمینه به دست آمده است. از آنجایی که اثر ریسمان باید با نقشه تابش زمینه گاوسی جمع شود لازم است که اثر دهانه یا بیم بر روی نقشه ریسمان‌ها نیز اعمال شود و نقشه ریسمان با همان بیم متناظر بخش گاوسی هموار شود. سه نقشه شبیه‌سازی شبکه ریسمانِ Ringeval با 
 $N_{\rm side} = 2048$
 و یک نقشه شبکه ریسمان با
 $N_{\rm side} = 4096$
 در پایگاه UCLouvain
 \LTRfootnote{\url{http://cp3.irmp.ucl.ac.be/~ringeval/upload/data/}}
  در دسترس است.
مقدار تنش ریسمان شدت شبکه ریسمان را مشخص می‌کند. هرچه تنش ریسمان بزرگتر باشد انحراف از گاوسی در CMB نمایان‌تر است و هر چه تنش کوچک‌تر باشد تشخیص ناگوسیت نیز سخت‌تر می‌شود.


\begin{figure}[h!]
	\centering
	\begin{subfigure}{\textwidth}
		\centering
		\includegraphics[scale=0.7]{figs/final1e-5.jpg}
		\caption{نقشه گاوسی-ریسمان با مقدار تنش نسبتا بزرگ
			($G\mu \sim 10^{-5}$)}
		%		\label{fig:sub1}
	\end{subfigure}%

	\begin{subfigure}{\textwidth}
		\centering
		\includegraphics[scale=0.7]{figs/final_1e-7.jpg}
		\caption{ نقشه گاوسی-ریسمان با مقدار تنش کوچک
		($G\mu \sim 10^{-8}$) }
		%		\label{fig:sub2}
	\end{subfigure}
	
	\caption{نمایش میزان انحراف از گاوسی بودن در نقشه‌های تابش زمینه در حضور ریسمان. شکل بالا با تنش ریسمان بزرگ و شکل پایین با تنش ریسمان کوچک. وقتی تنش کوچک باشد تشخیص ناگوسی بودن و ردپای ریسمان با چشم ممکن نیست.  شبیه‌سازی استفاده شده در این شکل 
		\lr{FFP10}
		بدون نوفه است. }
	\label{fig:tot}
\end{figure}
 شکل 
\ref{fig:tot}
دو نمونه از شبیه‌سازی CMB در حضور ریسمان را نشان می‌دهد. یکی با تنش بزرگ و دیگری با تنشی کوچک.
\begin{figure}[h!]
	\centering
	\begin{subfigure}{\textwidth}
		\centering
		\includegraphics[scale=0.3]{figs/step5_2.jpg}
		\caption{مقایسه طیف‌های توان شبیه‌سازی بدون ریسمان با شبیه‌سازی‌های حاوی  
			$G\mu$
		. همانطور که انتظار داریم اثرات ریسمان در مقیاس‌های کوچک که متناظر با 
	$\ell$
‌های بزرگ است، قابل توجه است و هرچه مقدار تنش ریسمان کوچک‌تر باشد این اثرات کم‌تر به چشم می‌آید.}
		%		\label{fig:sub1}
	\end{subfigure}%

	\begin{subfigure}{\textwidth}
		\centering
		\includegraphics[scale=0.8]{figs/pdf.png}
		\caption{ توزیع دمای نقشه‌های شبیه‌سازی. هیستوگرام نارنجی رنگ مربوط به شبیه‌سازی با 
			($G\mu = 10^{-6}$) 
		و توزیع آبی رنگ برای حالت بدون ریسمان است. همانطور که از این شکل برمی‌آید، تفاوت در توابع توزیع بسیار کم است و تشخیص انحراف از توزیع گاوسی از طریق بررسی توابع توزیع اثربخش نخواهد بود. به همین خاطر است که باید به دنبال ابزارهای قوی‌تری برای آشکارسازی ریسمان باشیم.   }
		%		\label{fig:sub2}
	\end{subfigure}
	
	\caption{مقایسه برخی خواص آماری نقشه‌های شبیه‌سازی در حضور ریسمان با حالت بدون ریسمان.  در شکل بالا طیف توان و در شکل پایین تابع توزیع نشان داده شده است.
	این شکل‌ها متعلق به آزمایش
\lr{E2E}
بدون نوفه است. }
	\label{fig:stat_comparison}
\end{figure}