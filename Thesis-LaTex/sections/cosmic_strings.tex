\begin{center}
	\begin{minipage}{\textwidth}
		\begin{traditionalpoem}
			هزار نعره ز بالای آسمان آمد،		& 
			
			تو تن زنی و نگویی این فغان ز کجا \\
			
			دلا دلا به سررشته شو مثل بشنو،		&
			
			که آسمان ز کجایست و ریسمان ز کجا \\
			\\
			& 
			\hspace{45pt}
			مولوی
		\end{traditionalpoem}
	\end{minipage}
\end{center}

پیش‌تر گفتیم که مدل ماده تاریک سرد به همراه ثابت کیهان‌شناسی به انضمام مدل تورمی بهترین مدل ارائه شده تاکنون برای توصیف کیهان و توجیه مشاهدات رصدی
\cite{ade2016planck}
است. یکی از پیش‌بینی‌های مدل‌های تورمی، رخ دادن گذار فازهایی در کیهان اولیه است که به دلیل انبساط کیهان و سرد شدن ناگهانی آن رخ می‌دهد. ریسمان‌های کیهانی، ساختارهایی خطی هستند که در اثر شکست تقارن خودبه‌خودی در اثر این گذار فازها در کیهان اولیه ممکن است شکل گرفته باشند.
\cite{kibble1976topology ,kibble1980some ,hindmarsh1995cosmic, copeland2010cosmic}
شبکه ریسمان‌های کیهانی به واسطه سیر تکامل غیرخطی خود می‌تواند ردّپاهای قابل‌توجهی در کیهان ایجاد کند.اگر اثرات آن‌ها دیده شود نه تنها وجود ریسمان‌های کیهانی تایید می‌شود، بلکه مطالعه خواص آن‌ها این امکان را به ما می‌دهد که بتوانیم درباره تاریخچه کیهان نظیر مدل‌های تورمی و شکست تقارن‌ها اطلاعاتی کسب کنیم. علاوه بر آن، مطالعه ریسمان‌ها در نظریات ورای مدل استاندارد ذرات بنیادی مانند نظریه وحدت بزرگ
\LTRfootnote{Grand Unified Theory}
نیز مفید خواهد بود. بنابراین انگیزه برای آشکارسازی ریسمان‌های کیهانی از این رو بیشتر می‌شود که مشاهده آن‌ها تاییدی بر نظریات کیهان‌شناسی و ذرات بنیادی است.  
\cite{kuroyanagi2013forecast}
ریسمان‌ها مانند خط و بسیار دراز هستند که پیش‌بینی می‌شود حدود ده‌ها ریسمان در افق هابلی وجود داشته باشند. این ریسمان‌ها انرژی زیادی در خود به دام می‌اندازند که مقدار شدت آن را با کمیتی بی بعد به نام تنش ریسمان پرمایش می‌کنیم. تنش ریسمان نشان‌گر جرم بر واحد طول آن است که به صورت زیر تعریف می‌شود:
\begin{equation}
\frac{G\mu}{c^2}=\order{\frac{\varpi^2}{M_{\rm Planck}^2}}.
\end{equation}
\label{eq:gmu}
که در آن
$\varpi$
مقیاس انرژی شکست تقارن است و 
$M_{\rm Planck}\equiv\sqrt{\hbar c/G}$
جرم پلانک است.
\cite{vafaei2017multiscale}
میزان تنش ریسمان‌ها رابطه نزدیکی با مقیاس انرژی‌ای که در آن شکست تقارن رخ می‌دهد دارد. گذاشتن حد بالا روی مقدار تنش ریسمان هدفی است که ما در این تحقیق پی گرفته‌ایم. اگر ریسمانی وجود داشته باشد ردپاهایی از خود به جا خواهد گذاشت که در این فصل این ردپاها را معرفی می‌کنیم تا در مباحث بعدی آن‌ها را دنبال کنیم تا شاید به سرنخی از وجود ریسمان در کیهان برسیم :)
 
\section{شکست خودبه‌خودی تقارن وتشکیل ریسمان‌ها}
\label{sec:ssb}
یکی از ویژگی‌های برخی سامانه‌های فیزیکی این است که تحول آن‌ها با گذار فاز همراه می‌شود. در مایعات، آب به عنوان مثال، هیچ جهت مرجحی وجود ندارد و سامانه دارای تقارن دورانی است. اما وقتی آب شروع به یخ زدن می‌کند کریستال حاصل جهات مرجح دارد و تقارن کلی شکسته می‌شود. گرچه در نقاط مختلفِ کریستال تقارن‌های موضعی وجود دارند ولی در این نقاط به سختی می‌توان از وجود یک تقارن بزرگ‌تر اطلاع پیدا کرد.
\cite{davis2005fundamental}
در اثر سرد شدن و رسیدن به نقطه انجماد، وقتی یک مایع در حال تبدیل شدن به کریستال جامد است یخ زدن را از مراکز کوچکی شروع می‌کند که نمی‌توان حدس زد جهت‌گیری آن به چه سمتی خواهد بود و انتخاب این نقاط مرکزی نیز تصادفی است و انتخاب‌های مختلف، نتایج متفاوتی خواهد داشت. در نهایت وقتی تمام مایع تبدیل به کریستال شود در اثر این پدیده شاهد عدم تطابق‌ها و ناسازگاری‌هایی در شبکه کریستال خواهیم بود که منجر به ایجاد مرزبندی و یا نقیصه‌هایی در آن می‌شود. مشابه این پدیده در کیهان اولیه و در اثر پایین آمدن دما رخ می‌دهد که منجر به تشکیل نقیصه‌های توپولوژیکی
\LTRfootnote{Topological defects} 
می‌شود. در سامانه‌هایی که شکست خودبه‌خودی تقارن اتفاق می‌افتد عموما این نقیصه‌ها به وجود می‌آیند به این دلیل که در این سامانه‌ها نیاز به انتخاب بین چندین حالت یکسان است و این انتخاب‌ها در نقاط مختلف سامانه متفاوت است.
\cite{rajantie2003defect}
بسته به نوع شکست تقارن، اقسام مختلفی از نقیصه‌های توپولوژیک می‌توانند تشکیل شده باشند. نقیصه‌های نقطه‌ای مانند تک‌قطبی
\LTRfootnote{Monopole}
، خطی مانند ریسمان‌های کیهانی و مسطح مانند دیوار حائل
\LTRfootnote{Domain wall}
. برای مطالعه انواع نقیصه‌های توپولوژیک می‌توان به مرجع
\cite{vilenkin2000cosmic}    
مراجعه کرد.
در میان این نقیصه‌ها، ریسمان‌های کیهانی از قضا جذاب‌ترند. در دهه‌های گذشته بسیاری از مطالعات به بررسی آثار و پیامدها و جستجو برای یافتن نقیصه‌های توپولوژیک اختصاص یافته‌اند. این مطالعات نشان می‌دهد که در میان انواع نقیصه‌ها، ریسمان‌های کیهانی جایگاه ممتازی از نظر سازگاری با مشاهدات دارا هستند. اما دیوارهای حائل و تک‌قطبی‌ها با مشکلات جدی کیهان‌شناسی روبرو هستند. تورم در آغاز کیهان می‌تواند مشکل تک‌قطبی را حل کند. اگر در ابتدای کیهان یک فاز انبساط تندشونده رخ دهد، این انبساط می‌تواند نقیصه‌های موجود را به قدری رقیق کند که اثر قابل ملاحظه‌ای از آن‌ها در کیهان باقی نماند.
\cite{guth1981inflationary}  
تورم با رقیق کردن نقیصه‌های توپولوژیک و تقلیل دادن تعداد آن‌ها تا مقدار یک عدد در شعاع هابلی می‌تواند مشکل تک‌قطبی را حل کند. البته این در صورتی اتفاق می‌افتد که نقیصه‌ها قبل دوران تورم تشکیل شده باشند.         
\cite{avelino2007effects}
برخلاف تک‌قطبی‌ها، ریسمان‌های کیهانی اما در پایان دوران تورم ایجاد شده‌ اند.
\cite{jeannerot2003generic}

\begin{figure}
	\begin{center}
		\includegraphics[scale=0.4]{figs/sombrero.png}
	\end{center}
	\caption[
	پتانسیل کلاه مکزیکی به عنوان تابعی از $\phi$. کمینه انرژی در حالتی است که
	$\abs{\phi} = \eta$
	و تمام نقاط واقع در دایره به شعاع $\eta$، حالت پایه انرژی هستند. شکل از مرجع 
	\cite{davis2005fundamental}]
	{  
		پتانسیل کلاه مکزیکی به عنوان تابعی از $\phi$. کمینه انرژی در حالتی است که
		$\abs{\phi} = \eta$
		و تمام نقاط واقع در دایره به شعاع $\eta$، حالت پایه انرژی هستند. شکل از مرجع 
		\cite{davis2005fundamental}
	}
	\label{fig:sombrero}
\end{figure}
\par

در ادامه ساده‌ترین مدلی که می‌تواند شکست تقارن خودبه‌خودی را توضیح دهد و بگوید که ریسمان‌ها چگونه اند را با تکیه بر مرجع
\cite{davis2005fundamental}
معرفی می‌کنیم. این مدل برای اولین بار توسط گلدستون
\LTRfootnote{Goldstone}
در سال ۱۹۶۱ مورد مطالعه قرار گرفت.
\cite{goldstone1961field}
ویژگی‌های اساسی یک شکست تقارن خودبه‌خودی توسط مدلی با یک میدان نرده‌ای 
$\phi$
توصیف می‌شود که به آن میدان هیگز نیز گفته می‌شود و می‌تواند کمیتی مختلط باشد، یعنی:
\begin{center}
$\phi = \phi_{1} +i\phi_{2}$
\end{center}
فرض می‌کنیم همیلتونیِ توصیف‌گر میدان تحت دوران فاز یعنی 
$ \phi \rightarrow \phi e^{i\alpha}$ 
ناوردا باشد. یا معادل آن:
\begin{equation}
\begin{split}
& \phi_1 \rightarrow \phi_1 \cos{\alpha} - \phi_2 \sin{\alpha} \\
& \phi_2 \rightarrow \phi_1 \sin{\alpha} + \phi_2 \cos{\alpha}
\end{split}
\end{equation}
انرژی پتانسیل V که فقط تابع
$\abs{\phi}$
است را به صورت زیر تعریف می‌کنیم 
\begin{equation}
V = \frac{1}{2} \lambda ( \abs{\phi}^{2} - \eta^{2}) = \frac{1}{2} \lambda ( \phi_1^{2} + \phi_2^{2} - \eta^{2})
\end{equation}
که به آن پتانسیل کلاه مکزیکی 
\LTRfootnote{Sombrero potential}
می‌گویند که $\eta$ در آن ثابت است.کمینه انرژی در
$\abs{\phi} = \eta$
و نه در صفر رخ می‌دهد. در شکل
\ref{fig:sombrero}
این پتانسیل نمایش داده شده است. تمام نقاط روی دایره به شعاع 
$\abs{\phi} = \eta$
 کمینه انرژی هستند یعنی در حالت پایه انرژی تبهگنی وجود دارد. تمام نقاط     
$\phi = \eta\exp{i\theta} $
معرف حالت پایه هستند و $\theta$ یک فاز دلخواه است، به عبارتی تقارن $U(1)$ موضعی وجود دارد. تقارن $U(1)$ به این معناست که لاگرانژی   تحت هر تبدیل $\phi \rightarrow \phi \exp{i\alpha} $ ناورداست و موضعی بودن این تقارن به معنی وابسته بودن $\alpha$ به فضا-زمان است.
لاگرانژی فرم کلاسیک زیر را داراست:
\begin{equation}
\cal L  = - \partial_{\mu} \phi^{\dagger}\space \partial^{\mu} \phi   - V(\phi)
\end{equation}

اگر بین همه انتخاب‌های موجود برای حالت خلاء یکی را انتخاب کنیم تقارن موضعی دیگر در سامانه وجود نخواهد و خودبه‌خود می‌شکند. شکست خودبه‌خودی تقارن موضعی سبب تولید ذره جرم‌دار و معرفی جرم به سامانه می‌شود  که به آن مکانیزم هیگز
\LTRfootnote{Higgs Mechanism}
می‌گویند.    
\cite{robinson2011symmetry , vilenkin2000cosmic}
در دمای بالا افت و خیزهای شدید وجود دارد که انرژی بالایی دارند. در این حالت می‌توان از برآمدگی در مرکز پتانسیل کلاه مکزیکی صرف‌نظر کرد و حالت پایه را در $\abs{\phi} = 0$ و سامانه را کاملا متقارن حول صفر در نظر گرفت. اما با پایین آمدن دما دیگر افت و خیز حول مقدار بیشنه‌ی مبداء امکان‌پذیر نیست و در نهایت میدان یکی از حالت‌های پایه انرژی را انتخاب می‌کند و در آن باقی می‌ماند. این که کدام نقطه از نقاط دایره برگزیده شود یک انتخاب تصادفی است و این انتخاب خودبه‌خودی تقارن سامانه را می‌شکند. وقتی یک سامانه وارد گذار فاز می‌شود، همانند مثال ما و یا مسئله کریستال آب، انتخاب‌ها در نقاط مختلف یکی نخواهد بود و نقاط دور از هم نیز نمی‌توانند از تصمیم یکدیگر آگاه شوند. انتخاب های تصادفی ممکن است متناسب با هم نباشند و منجر به تشکیل نقیصه‌های خطی یا ریسمان‌های کیهانی شود.
\cite{davis2005fundamental , perivolaropoulos2005rise}           
%---------------------------%---------------------------%---------------------------%---------------------------
\section{گرانش ریسمان و ردّپاهایش}
\label{sec:spacetime}
ریسمان‌های کیهانی اگر وجود داشته باشند می‌توانند ردّپاهایی از خود به جا بگذارند و ما به وسیله دنبال کردن این ردپاها می‌توانیم از وجود آن‌ها اطمینان پیدا کنیم و با مطالعه آن‌ها به اطلاعات ارزشمندی از کیهان اولیه دست پیدا کنیم. از آن‌جا که ریسمان‌‌ها انرژی زیادی در خود به دام انداخته اند انتظار می‌رود که اثرات گرانشی از خود به جا بگذارند. علاوه بر آن ایجاد افت وخیزهای دمایی بر تابش زمینه نیز از اثراتی است که مورد انتظار است. در این بخش به معرفی اجمالی این ردّپاها و به طور خاص‌تر اثر گات-کایزر-استبینز
‫‬‬\LTRfootnote{Gott-‪Kaiser-Stebbins}
که این تحقیق بر مبنای آن است می‌پردازیم.
\par
با وجود انرژی بسیار زیاد بر واحد طول ریسمان یا همان $\mu$ ، فضا-زمان اطراف یک ریسمان کیهانیِ صاف و مستقیم به صورت موضعی تخت است. معادله حالت ریسمان با فرض اینکه در راستای محور z قرار گرفته باشد به شکل
\begin{flalign}
\begin{aligned}
&  p_z = - \rho &\\
& p_x = p_y = 0& \\
\end{aligned}&&&
\end{flalign}
است. در حالت غیرنسبیتی 
$p_i << \rho$
و
 $\nabla^{2} \phi = 4\pi G\rho $
است. اما در حالت نسبیتی معادله پواسون نسبیتی برای ریسمان به شکل
\begin{flalign}
\nabla^{2} \phi = 4\pi G(\rho + p_x + p_y + p_z) = 0 
\end{flalign}
 خواهد شد که به این معنی است که چشمه گرانشی بر اطراف وارد نمی‌شود.
 \cite{vilenkin2000cosmic}
 اما این به این معنی نیست که ریسمان هیچ اثر گرانشی‌ای ندارد. ریسمان‌های در حال حرکت از قضا تاثیرات ویژه‌ای بر ماده اطراف و فوتون‌های CMB دارد. فضا-زمان اطراف یک ریسمانِ صاف و ثابت شکل ساده 
 \begin{equation}
 {ds}^2 =  {dt}^2 -  {dz}^2 - {dr}^2 - r^2 {d\theta}^2
 \end{equation}
 را به خود می‌گیرد. این متریک مشابه فضای مینکوفسکی در مختصات استوانه‌ای است. اما از آن‌جایی که مختصات سمتی در اینجا به یک بازه خاص محدود است، یعنی 
 $0 \leq \theta \leq 2\pi (1-4G\mu)$ 
 فضا-زمان به شکل مخروطی است. در واقع به اندازه 
\begin{center}
 $\Delta = 8\pi G\mu$
\end{center}
 کسر زاویه‌ای وجود دارد و به عبارتی یک گووه با پهنای $\Delta$ از فضا حذف شده است و ایجاد لبه شده است.
  \cite{turner1990early}
  %---------------------------
  \begin{figure}
  	\begin{center}
  		\includegraphics[scale=0.5]{figs/conicmetric.jpg}
  	\end{center}
  	\caption[
  	فضا-زمان مخروطی اطراف یک ریسمان کیهانی. ]
  	{  
  		فضا-زمان مخروطی اطراف یک ریسمان کیهانی.
  		\footnotemark
  	}
  	\label{fig:conic}
  \end{figure}
\LTRfootnotetext{\url{https://aether.lbl.gov/eunhwa_webpage_2/stringdynamics.html}}
  %---------------------------
شکل
\ref{fig:conic}

. به این دلیل است که ریسمان‌های کیهانی می‌توانند تصاویر دوتایی از کهکشان‌های دوردست درست کنند و یا ناپیوستگی‌های خط-مانند در تابش زمینه به وجود آورند.
  \cite{ade2014planck}
   زمانی که ریسمان در راستای دید حرکت می‌کند، فوتون‌های CMB  در راه رسیدن به ناظر دچار خیز می‌شوند که این پدیده باعث به وجود آمدن یک انتقال نسبی در دمای تابش زمینه می‌شود. فرض کنید ریسمان با سرعت $v_s$ در حال حرکت در راستای عمود بر خط دید ناظر است. در اثر هندسه مخروطی فضا-زمان، فوتونی که از جلوی ریسمان عبور می‌کند به نسبت فوتونی از پشت آن می‌آید دچار انتقال دوپلری خواهد شد. این اثر باید افت و خیز در دمای تابش زمینه است که به شکل
    \begin{equation}
    \frac{\delta T}{T} = 8 \pi G \mu v_s 
    \end{equation}
    بر CMB اثر می‌کند.  ویژگی متمایز کننده‌ الگوی ناهمسانگردی ناشی از ریسمان‌های کیهانی بر روی دمای تابش زمینه، لبه‌گون بودن این ناپیوستگی‌ها است.
  \cite{kaiser1984microwave}
  این ناپیوستگی‌های خط-مانند در دمای CMB به نام اثر \textbf{گات-کایزر-استبینز }شناخته می‌شود که زین پس این اثر را به اختصار با GKS نشان می‌دهیم. اثر GKS معرف ناگوسیت‌های ناشی از وجود ریسمان است. 
  \cite{gott1985gravitational}
یک مثال از ناپیوستگی خط-مانند در CMB در سمت راست شکل 
\ref{fig:discont}
آورده شده است.
  در دیدگاه تورمی افت و خیزهای اولیه که توسط نوسانات کوانتومی میدان به وجود می‌آیند، ذاتاً گاوسی تولید می‌شوند. با این حال ناگوسیت‌های غیرگاوسی نیز می‌توانند به علت وجود اثرات غیرخطی در حین دوران تورمی و یا در اثر جفت‌شدگی با سایر میدان‌ها به وجود آیند، البته شدت این ناگوسیت‌ها بسیار کم خواهد بود.
  \cite{sefusatti2007bispectrum}
  این نوع ناگوسیت‌ها از نوع افت و خیزهای اولیه هستند. ریسمان‌های کیهانی اما چشمه تولید ناگوسیت در تمام زمان‌ها هستند و از منظر  CMB سیگنال‌های مختلفی تولید می‌کنند.    
  \cite{ringeval2010cosmic}
 
     
%---------------------------  
%\begin{figure}
%	\begin{center}
%		\includegraphics[scale=0.35]{figs/discont.png}
%	\end{center}
%	\caption
%	{  
%   ناپیوستگی در دمای تابش زمینه کیهانی، به وجود آمده توسط ریسمان. شکل از منبع
%		\cite{ade2014planck}
%	}
%	\label{fig:discont}
%\end{figure} 
\begin{figure}[h!]
	\centering
	\begin{subfigure}[t]{.3\textwidth}
		\centering
		\includegraphics[scale=0.2]{figs/discont.png}
	\end{subfigure}%
	\begin{subfigure}{.7\textwidth}
		\centering
		\includegraphics[scale=0.28]{figs/str_fullsky.png}
	\end{subfigure}
	\caption{ نمای یک آسمان کامل ریسمان کیهانی، هموار شده با بیم ۵ دقیقه قوسی.  رنگ‌ها در این نقشه  به بازه افت و خیزهای 
		$(\Delta T/T)/(G\mu /c^2)$
		اشاره دارند.
		 تکه کوچک از آسمان که در سمت راست تصویر آمده است نشان دهنده ناپیوستگی ناشی از ریسمان بر محدوده کوچکی از آسمان و اثر GKS است.  شکل‌ها از منبع
		\cite{ade2014planck}	
	}
	\label{fig:discont}
\end{figure}
%---------------------------%---------------------------
از دیگر ردپاهایی که از ریسمان به جا می‌ماند می‌توان به اثرات هم‌گرایی گرانشی اشاره کرد. شکل
\ref{fig:lensing}
در میان تمام ردپاهایی که ریسمان‌های کیهانی می‌توانند از خود به جا بگذارند، اثر کایزر-استبینز که روی CMB ایجاد می‌کنند مورد توجه ما در این تحقیق است. اثرات خط-مانند ریسمان‌ها بر تابش زمینه در شکل
\ref{fig:str}
نشان داده شده است. دامنه اثرگذاری ریسمان‌ها روی تابش زمینه را با $G\mu$ که پیشتر با رابطه
\ref{eq:gmu}
معرفی شد مشخص می‌کنیم که $\mu$ چگالی خطی ریسمان و $G$ ثابت جهانی گرانش است.
\cite{bouchet1988patterns}  
%---------------------------
\begin{figure}[h!]
	\begin{center}
		\includegraphics[width=0.87\linewidth , height=0.3\textheight]{figs/str.png}
	\end{center}
	\caption[
	گرادیان شدت ناهمسانگردی‌های ناشی از ریسمان‌های کیهانی در نقشه دمایی تابش زمینه در سمت چپ. برای ایجاد تضاد بهتر مقیاس شکل‌ لگاریتمی شده است. و سمت راست گرادیان شدت ریسمان، هموار شده با بیم پلانک-مانند ۵ دقیقه قوسی. شکل‌ها به مقدار تنش ریسمان بهنجار شده اند.  
	\cite{ringeval2010cosmic}]
	{  
		گرادیان شدت ناهمسانگردی‌های ناشی از ریسمان‌های کیهانی در نقشه دمایی تابش زمینه در سمت چپ. برای ایجاد تضاد بهتر مقیاس شکل‌ لگاریتمی شده است. و سمت راست گرادیان شدت ریسمان، هموار شده با بیم پلانک-مانند ۵ دقیقه قوسی. شکل‌ها به مقدار تنش ریسمان بهنجار شده اند.     
		\cite{ringeval2010cosmic}
	}
	\label{fig:str}
\end{figure}
%---------------------------	
\begin{figure}[h!]
	\centering
	\begin{subfigure}{.5\textwidth}
		\centering
		\includegraphics[scale=0.25]{figs/lensing1.png}
	\end{subfigure}%
	\begin{subfigure}{.5\textwidth}
		\centering
		\includegraphics[scale=0.26]{figs/lensing2.png}
	\end{subfigure}
	\caption{دو مثال از اثر همگرایی گرانشی ریسمان بر روی چشمه نور گسترده و تشکیل لبه در تصویر. شکل از منبع 
		\cite{sazhin2010gravitational}	
	}
	\label{fig:lensing}
\end{figure}
 %---------------------------  %---------------------------       %---------------------------           

\section{مطالعات پیشین انجام شده در باب آشکارسازی ریسمان}
%\label{sec:string_sims}
تا کنون روش‌های متعددی برای آشکارسازی ریسمان‌های کیهانی پیشنهاد شده و تلاش‌هایی در این زمینه صورت گرفته است. عموم این تلاش‌ها در راستای مقید کردن $G\mu$ ریسمان است. اینکه ما بدانیم در چه بازه‌ای از $G\mu$ها به دنبال تنش ریسمان باشیم برای آشکارسازی آن مثمر ثمر است. اگر ریسمان‌ها در حدود انرژی GUT تشکیل شده باشد
 $G\mu \sim 10^{-6}$
خواهد بود.
\cite{firouzjahi2005brane , kibble1976topology}    
بیان این نکته حائز اهمیت است که نتایج تحقیق‌ها بسته به اینکه از چه پیش‌فرض‌هایی استفاده می‌کنیم، می‌تواند متغیر باشد. برای شبیه‌سازی‌های ریسمان فرض‌های متفاوتی وجود دارد. به طور مثال دو نوع از شبیه‌سازی ریسمان نامبو-گاتو
\LTRfootnote{Nambu-Goto}
\cite{fraisse2008small}
و آبلین-هیگز
\LTRfootnote{Abelian-Higgs}
\cite{moore2001evolution , bevis2007cmb}
هستند که در کنش متفاوت اند. شبیه‌سازی‌های تابش زمینه نیز از حیث وضوح، اثر دهانه تلسکوپ و اثرات نوفه با یکدیگر متفاوت اند و شبیه‌سازی‌های مختلفی برای CMB  وجود دارد که در فصل
\ref{ch:simulations} 
درباره آن‌ها بیشتر می‌خوانیم.
\\
از منظر بررسی امواج گرانشی گسیل شده از ریسمان‌های نامبو-گاتو تنش ریسمان باید در بازه محدود
 
 $10^{-14} \leq G\mu \leq 10^{-10}$ 
 باشد. 
 \cite{ringeval2017stochastic}
از منظر مطالعه اثر کایزر-استبینز، در مقاله 
\cite{vafaei2017multiscale}
مقدار  
$G\mu \leq 1.2 \times 10^{-7}$
برای تنش ریسمان با استفاده از پردازش تصویر و مطالعات آماری و در مقاله 
\cite{vafaei2018cosmic}
با استفاده از روش‌های یادگیری ماشین درختی مقدار 
$G\mu \leq 3.0 \times 10^{-8}$
برای شبیه‌سازی‌های نسل چهارم تابش زمینه گزارش شده است. 
ریسمان‌ها بر روی تابش زمینه تاثیرات گوناگونی دارند. از جمله این تاثیرات اثر سکس-ولف عادی و تجمعی، اثر قطبش
\cite{baumann1980method}
، هم‌گرایی گرانشی و تاثیرات بر روی طیف توان است. مدل استاندارد کیهان‌شناسی با ۶ درجه آزادی توصیف می‌شود. برای در نظر گرفتن اثر ریسمان‌های کیهانی، می‌توانیم پارامتر هفتمی را نیز وارد مدل کنیم و بهترین برازش را برای ۷ درجه آزادی به دست آوریم. این پارامتر اضافه‌شده     
$f_{10}$
است که نشان‌دهنده طیف توان ریسمان به طیف توان تابش زمینه برای 
$l = 10$
است. بررسی طیف تابش زمینه نیز برای داده‌های ماهواره پلانک مطابق با 
$f_{10} \leq 0.015 $
که معادل 
$G\mu \leq 1.5 \times 10^{-7}$
برای مدل ریسمان نامبو-گاتو و 
$f_{10} \leq 0.033 $
یا 
$G\mu \leq 3.6 \times 10^{-7}$
برای مدل آبلین-هیگز است. 
 
\cite{ade2014planck}
جدیدترین مطالعه صورت گرفته برای آشکارسازی ریسمان از طریق بررسی ناهمسانگردی تابش زمینه، پژوهشی است که با استفاده از شبکه‌های عصبی صورت گرفته است. تنش گزارش شده توسط این روش از طریق اشاره به محل قرارگیری ریسمان، مقدار 
$G\mu \leq 2.3 \times 10^{-9}$
برای شبیه‌سازی‌های بدون نوفه و ایده‌آل است. 
\cite{ciuca2017bayesian, ciuca2019inferring}
برای داده‌های نوفه‌دار، با نوفه شبیه داده‌های تابش زمینه تلسکوپ آتاکاما، حد 
$G\mu \leq 1.2 \times 10^{-7}$
توسط الگوریتم‌های شبکه عصبی پیشنهاد شده است.
\cite{ciuca2020information}

\par
در فصل بعد ( فصل
\ref{ch:deep_learning}
)، از علم داده و یادگیری ماشین سخن می‌گوییم و انگیزه خود برای استفاده از آن‌‌ها را در کیهان‌شناسی بیان می‌کنیم. در نهایت‌ ابزار استفاده شده در این تحقیق، یعنی شبکه‌های عصبی پیچشی را معرفی خواهیم کرد تا بتوانیم در فصل‌های بعد به سراغ شرح روش این پژوهش برویم. 



  